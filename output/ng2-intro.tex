\documentclass[12pt,]{article}
\usepackage{lmodern}
\usepackage{amssymb,amsmath}
\usepackage{ifxetex,ifluatex}
\usepackage{fixltx2e} % provides \textsubscript
\ifnum 0\ifxetex 1\fi\ifluatex 1\fi=0 % if pdftex
  \usepackage[T1]{fontenc}
  \usepackage[utf8]{inputenc}
\else % if luatex or xelatex
  \ifxetex
    \usepackage{mathspec}
    \usepackage{xltxtra,xunicode}
  \else
    \usepackage{fontspec}
  \fi
  \defaultfontfeatures{Mapping=tex-text,Scale=MatchLowercase}
  \newcommand{\euro}{€}
    \setmainfont{Palatino}
    \setsansfont{Century Gothic}
    \setmonofont[Mapping=tex-ansi]{Consolas}
\fi
% use upquote if available, for straight quotes in verbatim environments
\IfFileExists{upquote.sty}{\usepackage{upquote}}{}
% use microtype if available
\IfFileExists{microtype.sty}{%
\usepackage{microtype}
\UseMicrotypeSet[protrusion]{basicmath} % disable protrusion for tt fonts
}{}
\ifxetex
  \usepackage[setpagesize=false, % page size defined by xetex
              unicode=false, % unicode breaks when used with xetex
              xetex]{hyperref}
\else
  \usepackage[unicode=true]{hyperref}
\fi
\hypersetup{breaklinks=true,
            bookmarks=true,
            pdfauthor={Amin Meyghani},
            pdftitle={Introduction to Angular 2},
            colorlinks=true,
            citecolor=blue,
            urlcolor=blue,
            linkcolor=magenta,
            pdfborder={0 0 0}}
\urlstyle{same}  % don't use monospace font for urls
\usepackage{fancyhdr}
\pagestyle{fancy}
\pagenumbering{arabic}
\lhead{\itshape Introduction to Angular 2}
\chead{}
\rhead{\itshape{\nouppercase{\leftmark}}}
\lfoot{}
\cfoot{}
\rfoot{\thepage}
\usepackage{color}
\usepackage{fancyvrb}
\newcommand{\VerbBar}{|}
\newcommand{\VERB}{\Verb[commandchars=\\\{\}]}
\DefineVerbatimEnvironment{Highlighting}{Verbatim}{commandchars=\\\{\}}
% Add ',fontsize=\small' for more characters per line
\newenvironment{Shaded}{}{}
\newcommand{\KeywordTok}[1]{\textcolor[rgb]{0.00,0.00,1.00}{{#1}}}
\newcommand{\DataTypeTok}[1]{{#1}}
\newcommand{\DecValTok}[1]{{#1}}
\newcommand{\BaseNTok}[1]{{#1}}
\newcommand{\FloatTok}[1]{{#1}}
\newcommand{\ConstantTok}[1]{{#1}}
\newcommand{\CharTok}[1]{\textcolor[rgb]{0.00,0.50,0.50}{{#1}}}
\newcommand{\SpecialCharTok}[1]{\textcolor[rgb]{0.00,0.50,0.50}{{#1}}}
\newcommand{\StringTok}[1]{\textcolor[rgb]{0.00,0.50,0.50}{{#1}}}
\newcommand{\VerbatimStringTok}[1]{\textcolor[rgb]{0.00,0.50,0.50}{{#1}}}
\newcommand{\SpecialStringTok}[1]{\textcolor[rgb]{0.00,0.50,0.50}{{#1}}}
\newcommand{\ImportTok}[1]{{#1}}
\newcommand{\CommentTok}[1]{\textcolor[rgb]{0.00,0.50,0.00}{{#1}}}
\newcommand{\DocumentationTok}[1]{\textcolor[rgb]{0.00,0.50,0.00}{{#1}}}
\newcommand{\AnnotationTok}[1]{\textcolor[rgb]{0.00,0.50,0.00}{{#1}}}
\newcommand{\CommentVarTok}[1]{\textcolor[rgb]{0.00,0.50,0.00}{{#1}}}
\newcommand{\OtherTok}[1]{\textcolor[rgb]{1.00,0.25,0.00}{{#1}}}
\newcommand{\FunctionTok}[1]{{#1}}
\newcommand{\VariableTok}[1]{{#1}}
\newcommand{\ControlFlowTok}[1]{\textcolor[rgb]{0.00,0.00,1.00}{{#1}}}
\newcommand{\OperatorTok}[1]{{#1}}
\newcommand{\BuiltInTok}[1]{{#1}}
\newcommand{\ExtensionTok}[1]{{#1}}
\newcommand{\PreprocessorTok}[1]{\textcolor[rgb]{1.00,0.25,0.00}{{#1}}}
\newcommand{\AttributeTok}[1]{{#1}}
\newcommand{\RegionMarkerTok}[1]{{#1}}
\newcommand{\InformationTok}[1]{\textcolor[rgb]{0.00,0.50,0.00}{{#1}}}
\newcommand{\WarningTok}[1]{\textcolor[rgb]{0.00,0.50,0.00}{\textbf{{#1}}}}
\newcommand{\AlertTok}[1]{\textcolor[rgb]{1.00,0.00,0.00}{{#1}}}
\newcommand{\ErrorTok}[1]{\textcolor[rgb]{1.00,0.00,0.00}{\textbf{{#1}}}}
\newcommand{\NormalTok}[1]{{#1}}
\setlength{\parindent}{0pt}
\setlength{\parskip}{6pt plus 2pt minus 1pt}
\setlength{\emergencystretch}{3em}  % prevent overfull lines
\providecommand{\tightlist}{%
  \setlength{\itemsep}{0pt}\setlength{\parskip}{0pt}}
\setcounter{secnumdepth}{5}

\title{Introduction to Angular 2}
\author{Amin Meyghani}
\date{}

% Redefines (sub)paragraphs to behave more like sections
\ifx\paragraph\undefined\else
\let\oldparagraph\paragraph
\renewcommand{\paragraph}[1]{\oldparagraph{#1}\mbox{}}
\fi
\ifx\subparagraph\undefined\else
\let\oldsubparagraph\subparagraph
\renewcommand{\subparagraph}[1]{\oldsubparagraph{#1}\mbox{}}
\fi

\begin{document}
\maketitle

{
\hypersetup{linkcolor=black}
\setcounter{tocdepth}{5}
\tableofcontents
}
\section{Installing Node}\label{installing-node}

\begin{itemize}
\tightlist
\item
  Use \href{https://github.com/creationix/nvm}{nvm} to install and
  manage Node on the machine. Copy the install script and run it:
\end{itemize}

\begin{Shaded}
\begin{Highlighting}[numbers=left,,]
\KeywordTok{curl} \NormalTok{-o- https://raw.githubusercontent.com/creationix/nvm/v0.30.1/install.sh }\KeywordTok{|} \KeywordTok{bash}
\end{Highlighting}
\end{Shaded}

\begin{itemize}
\tightlist
\item
  After installed, make sure that it is installed, by running:
\end{itemize}

\begin{Shaded}
\begin{Highlighting}[numbers=left,,]
\KeywordTok{nvm} \NormalTok{--help}
\end{Highlighting}
\end{Shaded}

\begin{itemize}
\tightlist
\item
  Then use \texttt{nvm} to install node version \texttt{0.12.9} by
  running:
\end{itemize}

\begin{Shaded}
\begin{Highlighting}[numbers=left,,]
\KeywordTok{nvm} \NormalTok{install 0.12.9}
\end{Highlighting}
\end{Shaded}

\begin{itemize}
\item
  Confirm that it is installed by running \texttt{node\ -v}.
\item
  You can load any node version in the current shell with
  \texttt{nvm\ use\ 0.x.y} after of course installing it.
\item
  You can make \texttt{0.12.9} the default version by making an alias
  for the default node:

\begin{Shaded}
\begin{Highlighting}[numbers=left,,]
\KeywordTok{nvm} \NormalTok{alias default 0.12.9}
\end{Highlighting}
\end{Shaded}

  \subsection{Permissions}\label{permissions}
\item
  Never use \texttt{sudo} to install packages, never do
  \texttt{sudo\ npm\ install\ \textless{}package\textgreater{}}. If you
  get permission errors, you can own the folders by the current user. So
  for example, if you get an error like:

\begin{Shaded}
\begin{Highlighting}[numbers=left,,]
\KeywordTok{Error}\NormalTok{: EACCES, mkdir }\StringTok{'/usr/local'}
\end{Highlighting}
\end{Shaded}
\item
  you can own the folder with:

\begin{Shaded}
\begin{Highlighting}[numbers=left,,]
\KeywordTok{sudo} \NormalTok{chown -R }\KeywordTok{`whoami`} \NormalTok{/usr/local}
\end{Highlighting}
\end{Shaded}

  You can own folders until node doesn't complain.
\end{itemize}

\subsection{Install a Package}\label{install-a-package}

\begin{itemize}
\item
  Install a package to verify that node is installed and everything is
  wired up correctly. We are going to use \texttt{live-server} through
  the course. So let's install that:

\begin{Shaded}
\begin{Highlighting}[numbers=left,,]
\KeywordTok{npm} \NormalTok{i -g live-server}
\end{Highlighting}
\end{Shaded}
\item
  Then, you should be able to run \texttt{live-server} in any folder to
  server the content of that folder:

\begin{Shaded}
\begin{Highlighting}[numbers=left,,]
\KeywordTok{mdkir} \NormalTok{~/Desktop/sample }\KeywordTok{&&} \KeywordTok{cd} \OtherTok{$_}
\KeywordTok{live-server} \NormalTok{.}
\end{Highlighting}
\end{Shaded}
\end{itemize}

\section{Visual Studio Code}\label{visual-studio-code}

\begin{itemize}
\item
  Install Visual Studio Code from: \url{https://code.visualstudio.com/}
\item
  You can open new projects by going to the
  \texttt{File\ \textgreater{}\ Open} tag, to etierh open a folder
  containing your project or a single file
\item
  Some useful keyboard shortcuts are:

  \begin{itemize}
  \tightlist
  \item
    \texttt{command\ +\ b}: to close/open the file navigator
  \item
    \texttt{command\ +\ shift\ +\ p}: to open the prompt
  \end{itemize}
\item
  To install extensions open the prompt with
  \texttt{command\ +\ shift\ +\ p} and type:

  \begin{itemize}
  \tightlist
  \item
    \texttt{\textgreater{}\ install\ extension}
  \end{itemize}
\item
  Open the shortcuts settings from
  \texttt{Preferences\ \textgreater{}\ Keyboard\ Shortcuts}, and then
  you can add your own shortcuts:

\begin{verbatim}
// Place your key bindings in this file to overwrite the defaults
[
  {
    "key": "cmd+t",
    "command": "workbench.action.quickOpen"
  },
  {
    "key": "shift+cmd+r",
    "command": "editor.action.format",
    "when": "editorTextFocus"
  }
]
\end{verbatim}
\end{itemize}

\section{TypeScript Crash-course}\label{typescript-crash-course}

\subsection{Installing TypeScript}\label{installing-typescript}

You can install the TypeScript compiler with node:

\begin{Shaded}
\begin{Highlighting}[numbers=left,,]
\KeywordTok{npm} \NormalTok{i typescript -g}
\end{Highlighting}
\end{Shaded}

Then to verify that it is installed, run \texttt{tsc\ -v} to see the
version of the compiler. You will get an output like this:

\begin{verbatim}
message TS6029: Version 1.7.5
\end{verbatim}

In addition to the compiler, we also need to install the TypeScript
Definition manager for DefinitelyTyped (tsd). You can install tsd with:

\begin{Shaded}
\begin{Highlighting}[numbers=left,,]
\KeywordTok{npm} \NormalTok{i tsd -g}
\end{Highlighting}
\end{Shaded}

Using TSD, you can search and install TypeScript definition files
directly from the community driven DefinitelyTyped repository. To verify
that tsd is installed, run tsd with the \texttt{version} flag:

\begin{Shaded}
\begin{Highlighting}[numbers=left,,]
\KeywordTok{tsd} \NormalTok{--version}
\end{Highlighting}
\end{Shaded}

You should get an output like this:

\begin{verbatim}
>> tsd 0.6.5
\end{verbatim}

After \texttt{tsd} and \texttt{tsc} are installed, we can compile a
hello world program:

\begin{itemize}
\item
  make a file called \texttt{hello.ts} on your desktop:

\begin{Shaded}
\begin{Highlighting}[numbers=left,,]
\KeywordTok{touch} \NormalTok{~/Desktop/hello.ts}
\end{Highlighting}
\end{Shaded}
\item
  Then, put some TypeScript code in the file:

\begin{Shaded}
\begin{Highlighting}[numbers=left,,]
\KeywordTok{echo} \StringTok{"const adder = (a: number, b: number): number => a + b;"} \KeywordTok{>} \NormalTok{~/Desktop/hello.ts}
\end{Highlighting}
\end{Shaded}
\item
  Then you can compile the file to JavaScript:

\begin{Shaded}
\begin{Highlighting}[numbers=left,,]
\KeywordTok{tsc} \NormalTok{~/Desktop/hello.ts}
\end{Highlighting}
\end{Shaded}
\item
  It should output a file in \texttt{Desktop/hello.js}:

\begin{Shaded}
\begin{Highlighting}[numbers=left,,]
\KeywordTok{var} \NormalTok{adder }\OperatorTok{=} \KeywordTok{function} \NormalTok{(a}\OperatorTok{,} \NormalTok{b) }\OperatorTok{\{} \ControlFlowTok{return} \NormalTok{a }\OperatorTok{+} \NormalTok{b}\OperatorTok{;} \OperatorTok{\};}
\end{Highlighting}
\end{Shaded}
\end{itemize}

Now that your TypeScript compiler setup, we can move on to configuring
Visual Studio Code.

\subsection{Setting up TypeScript for
VSCode}\label{setting-up-typescript-for-vscode}

You can set up Visual Studio Code to compile your TypeScript code as
your work.

\begin{itemize}
\item
  First, open Visual Studio Code
\item
  Make a new window: \texttt{File\ \textgreater{}\ New\ Window}
\item
  Then, make a folder on your desktop for a new project:
  \texttt{mkdir\ \textasciitilde{}/Desktop/vscode-demo}
\item
  The, open the folder in VSCode: \texttt{File\ \textgreater{}\ open}
  and select the \texttt{vscode-demo} folder on your desktop.
\item
  Now we need to make three configuration files:

  \begin{enumerate}
  \def\labelenumi{\arabic{enumi}.}
  \tightlist
  \item
    \texttt{tsconfig.json}: configuration for the TypeScript compiler
  \item
    \texttt{tasks.json}: Task configuration for VSCode to watch and
    compile files
  \item
    \texttt{launch.json}: Configuration for the debugger
  \end{enumerate}
\item
  The \texttt{tsconfig.json} file should be in the root of the project.
  Let's make the file and put the following in it:

\begin{verbatim}
{
  "compilerOptions": {
    "experimentalDecorators": true,
    "emitDecoratorMetadata": true,
    "module": "commonjs",
    "target": "es5",
    "sourceMap": true,
    "outDir": "output",
    "watch": true
  }
}
\end{verbatim}
\item
  Now to make the \texttt{tasks.json} file, open the prompt with
  \texttt{command\ +\ shift\ +\ p} and type:

\begin{verbatim}
> configure task runner
\end{verbatim}
\item
  Then put the following in the file and save the file:

\begin{verbatim}
{
  "version": "0.1.0",
  "command": "tsc",
  "showOutput": "silent",
  "isShellCommand": true,
  "problemMatcher": "$tsc"
}
\end{verbatim}
\item
  The last thing that we need to set up is the debugger,
  i.e.\texttt{launch.json} file. Right click on the \texttt{.vscode}
  folder in the file navigator and make a new file called
  \texttt{launch.json} and put in the following:

\begin{verbatim}
{
  "version": "0.1.0",
  "configurations": [
    {
      "name": "TS Debugger",
      "type": "node",
      "program": "main.ts",
      "stopOnEntry": false,
      "sourceMaps": true,
      "outDir": "output"
    }
  ]
}
\end{verbatim}
\item
  After you save the file, you should be able to see the debugger in the
  debugger dropdown options.
\item
  Now, we are ready to make the \texttt{main.ts} file in the root of the
  project:

  \textbf{\texttt{main.ts}}

\begin{Shaded}
\begin{Highlighting}[numbers=left,,]
\NormalTok{console.}\FunctionTok{log}\NormalTok{('hello');}
\end{Highlighting}
\end{Shaded}
\item
  Now you can start the task to watch the files and compile as you work.
  Open the prompt with \texttt{command\ +\ shift\ +\ p} and type:

\begin{verbatim}
> run build tasks
\end{verbatim}

  you can also use the \texttt{command\ +\ shift\ +\ b} keyboard
  shortcut instead. This will start the debugger and watch the files.
  After making a change to \texttt{main.ts}, you should be able to see
  the output in the \texttt{output} folder.
\item
  Now that the build task is running, we can put a breakpoint anywhere
  in our typescript code. Lets add some more code to the main file and
  use the debugger:

\begin{Shaded}
\begin{Highlighting}[numbers=left,,]
\NormalTok{let a = }\DecValTok{2}\NormalTok{;}
\NormalTok{let b = }\DecValTok{3}\NormalTok{;}
\NormalTok{let c = }\DecValTok{4}\NormalTok{;}
\end{Highlighting}
\end{Shaded}
\item
  Then click on the margin of line two for example to add a breakpoint.
  Then open the debugger tab to run the debugger and you should see that
  the program will stop at the breakpoint and you can step over or into
  the line.
\item
  To stop the task you can terminate it. Open the prompt and type:

\begin{verbatim}
> terminate running task
\end{verbatim}
\end{itemize}

\subsection{Types and the Basics}\label{types-and-the-basics}

There are 7 types in TypeScript:

\begin{itemize}
\tightlist
\item
  boolean: \texttt{var\ isDone:\ boolean\ =\ false;}
\item
  number: \texttt{var\ height:\ number\ =\ 6;}
\item
  string: \texttt{var\ name:\ string\ =\ "bob";}
\item
  array: \texttt{var\ list:number{[}{]}\ =\ {[}1,\ 2,\ 3{]};} also
  \texttt{var\ list:Array\textless{}number\textgreater{}\ =\ {[}1,\ 2,\ 3{]};}
\item
  enum: \texttt{enum\ Color\ \{Red,\ Green,\ Blue\};}
\item
  any: \texttt{var\ notSure:\ any\ =\ 4;}
\item
  void:
  \texttt{function\ hello():\ void\ \{\ console.log(\textquotesingle{}hello\textquotesingle{});\ \}}
\end{itemize}

\subsection{Interface}\label{interface}

\begin{itemize}
\tightlist
\item
  An Interface is defined using the \texttt{interface} keyword
\item
  Interfaces are used only during compilation time to check types
\item
  By convention, interface definitions start with an \texttt{I}, e.g. :
  \texttt{IPoint}
\item
  Interfaces are used in classical object oriented programming as a
  design tool
\item
  Interfaces don't contain implementations
\item
  They provide definitions only
\item
  When an object implements an interface, it must adhere to the contract
  defined by the interface
\item
  An interface defines what properties and methods an object must
  implement
\item
  If an object implements an interface, it must adhere to the contract.
  If it doesn't the compiler will let us know.
\item
  Interfaces also define custom types
\end{itemize}

\subsubsection{Basic Interface}\label{basic-interface}

Below is an example of an Interface that defines two properties and
three methods that implementers should provide implementations for:

\begin{Shaded}
\begin{Highlighting}[numbers=left,,]
\KeywordTok{interface} \NormalTok{IMyInterface \{}
  \CommentTok{// some properties}
  \NormalTok{id: number;}
  \NormalTok{name: string;}

  \CommentTok{// some methods}
  \FunctionTok{method}\NormalTok{(): }\DataTypeTok{void}\NormalTok{;}
  \FunctionTok{methodWithReturnVal}\NormalTok{():number;}
  \FunctionTok{sum}\NormalTok{(nums: number[]):number;}
\NormalTok{\}}
\end{Highlighting}
\end{Shaded}

Using the interface above we can create an object that adheres to the
interface:

\begin{Shaded}
\begin{Highlighting}[numbers=left,,]
\NormalTok{let myObj: IMyInterface = \{}
  \NormalTok{id: }\DecValTok{2}\NormalTok{,}
  \NormalTok{name: 'some name',}

  \FunctionTok{method}\NormalTok{() \{ console.}\FunctionTok{log}\NormalTok{('hello'); \},}
  \FunctionTok{methodWithReturnVal} \NormalTok{() \{ }\KeywordTok{return} \DecValTok{2}\NormalTok{; \},}
  \FunctionTok{sum}\NormalTok{(numbers) \{}
    \KeywordTok{return} \NormalTok{numbers.}\FunctionTok{reduce}\NormalTok{( (a,b) => \{ }\KeywordTok{return} \NormalTok{a + b \} );}
  \NormalTok{\}}
\NormalTok{\};}
\end{Highlighting}
\end{Shaded}

Notice that we had to provide values to \textbf{all} the properties
defined by the Interface, and the implementations for \textbf{all} the
methods defined by the Interface.

And then of course you can use your object methods to perform
operations:

\begin{Shaded}
\begin{Highlighting}[numbers=left,,]
\NormalTok{let sum = myObj.}\FunctionTok{sum}\NormalTok{([}\DecValTok{1}\NormalTok{,}\DecValTok{2}\NormalTok{,}\DecValTok{3}\NormalTok{,}\DecValTok{4}\NormalTok{,}\DecValTok{5}\NormalTok{]); }\CommentTok{// -> 15}
\end{Highlighting}
\end{Shaded}

\subsection{Classes}\label{classes}

\begin{itemize}
\tightlist
\item
  Classes are heavily used in classical object oriented programming
\item
  It defines what an object is and what it can do
\item
  A class is defined using the \texttt{class} keyword followed by a name
\item
  By convention, the name of the class start with an uppercase letter
\item
  A class can be used to create multiple objects (instances) of the same
  class
\item
  An object is created from a class using the \texttt{new} keyword
\item
  A class can have a \texttt{constructor} which is called when an object
  is made from the class
\item
  Properties of a class are called instance variables and its functions
  are called the class methods
\item
  Access modifiers can be used to make them public or private
\item
  The instance variables are attached to the instance itself but not the
  prototype
\item
  Methods however are attached to the prototype object as opposed to the
  instance itself
\item
  Classes can inherit functionality from other classes, but you should
  \href{https://medium.com/javascript-scene/the-two-pillars-of-javascript-ee6f3281e7f3\#.oc5pdevwh}{favor
  composition over inheritance} or make sure you know
  \href{https://medium.com/@dtinth/es6-class-classical-inheritance-20f4726f4c4\#.xdif2m42e}{when
  to use it}
\item
  Classes can implement interfaces
\end{itemize}

Let's make a class definition for a car and incrementally add more
things to it.

\subsubsection{Adding an Instance
Variable}\label{adding-an-instance-variable}

The \texttt{Car} class definition can be very simple and can define only
a single instance variable that all cars can have:

\begin{Shaded}
\begin{Highlighting}[numbers=left,,]
\KeywordTok{class} \NormalTok{Car \{}
  \NormalTok{distance: number;}
\NormalTok{\}}
\end{Highlighting}
\end{Shaded}

\begin{itemize}
\tightlist
\item
  \texttt{Car} is the name of the class, which also defines the custom
  type \texttt{Car}
\item
  \texttt{distance} is a property that tracks the distance that car has
  traveled
\item
  Distance is of type \texttt{number} and only accepts \texttt{number}
  type.
\end{itemize}

Now that we have the definition for a car, we can create a car from the
definition:

\begin{Shaded}
\begin{Highlighting}[numbers=left,,]
\NormalTok{let myCar:Car = }\KeywordTok{new} \FunctionTok{Car}\NormalTok{();}
\NormalTok{myCar.}\FunctionTok{distance} \NormalTok{= }\DecValTok{0}\NormalTok{;}
\end{Highlighting}
\end{Shaded}

\begin{itemize}
\tightlist
\item
  \texttt{myCar:Car} means that \texttt{myCar} is of type \texttt{Car}
\item
  \texttt{new\ Car()} creates an instance from the \texttt{Car}
  definition.
\item
  \texttt{myCar.distance\ =\ 0} sets the initial value of the
  \texttt{distance} to 0 for the newly created \texttt{car}
\end{itemize}

\subsubsection{Adding a Method}\label{adding-a-method}

So far our car doesn't have any definitions for any actions. Let's
define a \texttt{move} method that all the cars can have:

\begin{Shaded}
\begin{Highlighting}[numbers=left,,]
\KeywordTok{class} \NormalTok{Car \{}
  \NormalTok{distance: number;}
  \FunctionTok{move}\NormalTok{():}\DataTypeTok{void} \NormalTok{\{}
    \KeywordTok{this}\NormalTok{.}\FunctionTok{distance} \NormalTok{+= }\DecValTok{1}\NormalTok{;}
  \NormalTok{\};}
\NormalTok{\}}
\end{Highlighting}
\end{Shaded}

\begin{itemize}
\tightlist
\item
  \texttt{move():void} means that \texttt{move} is a method that does
  not return any value, hence \texttt{void}.
\item
  The body of the method is defined in \texttt{\{\ \}}
\item
  \texttt{this} refers to the instance, therefore \texttt{this.distance}
  points to the \texttt{distance} property defined on the car instance.
\item
  Now you can call the \texttt{move} method on the car instance to
  increment the \texttt{distance} value by 1:
\end{itemize}

\begin{Shaded}
\begin{Highlighting}[numbers=left,,]
\NormalTok{myCar.}\FunctionTok{move}\NormalTok{();}
\NormalTok{console.}\FunctionTok{log}\NormalTok{(myCar.}\FunctionTok{distance}\NormalTok{) }\CommentTok{// -> 1}
\end{Highlighting}
\end{Shaded}

\subsubsection{Adding a constructor}\label{adding-a-constructor}

A \texttt{constructor} is a special method that gets called when an
instance is created from a class. Let's add a constructor to the
\texttt{Car} class that initializes the \texttt{distance} value to 0.
This means that all the cars that are crated from this class, will have
their \texttt{distance} set to 0 automatically:

\begin{Shaded}
\begin{Highlighting}[numbers=left,,]
\KeywordTok{class} \NormalTok{Car \{}
  \NormalTok{distance: number;}
  \FunctionTok{constructor} \NormalTok{() \{}
    \KeywordTok{this}\NormalTok{.}\FunctionTok{distance} \NormalTok{= }\DecValTok{0}\NormalTok{;}
  \NormalTok{\};}
  \FunctionTok{move}\NormalTok{():}\DataTypeTok{void} \NormalTok{\{}
    \KeywordTok{this}\NormalTok{.}\FunctionTok{distance} \NormalTok{+= }\DecValTok{1}\NormalTok{;}
  \NormalTok{\};}
\NormalTok{\}}
\end{Highlighting}
\end{Shaded}

\begin{itemize}
\tightlist
\item
  \texttt{constructor()} is called automatically when a new car is
  created
\item
  The body of the constructor is defined in the \texttt{\{\ \}}
\end{itemize}

So now when we create a car, the \texttt{distance} property is
automatically set to 0.

\subsubsection{Using Access Modifiers}\label{using-access-modifiers}

If you wanted to tell the compiler that the \texttt{distance} variable
is private and can only be used by the object itself, you can use the
\texttt{private} modifier before the name of the property:

\begin{Shaded}
\begin{Highlighting}[numbers=left,,]
\KeywordTok{class} \NormalTok{Car \{}
  \KeywordTok{private} \NormalTok{distance: number;}
  \FunctionTok{constructor} \NormalTok{() \{}
    \NormalTok{...}
  \NormalTok{\};}
  \NormalTok{...}
\NormalTok{\}}
\end{Highlighting}
\end{Shaded}

Access modifiers can be used in different places. Check out the access
modifiers chapter for more details.

\subsubsection{Implementing an
Interface}\label{implementing-an-interface}

Classes can implement one or multiple interfaces. We can make the
\texttt{Car} class implement two interfaces:

\textbf{interfaces}

\begin{Shaded}
\begin{Highlighting}[numbers=left,,]
\KeywordTok{interface} \NormalTok{ICarProps \{}
  \NormalTok{distance: number;}
\NormalTok{\}}
\KeywordTok{interface} \NormalTok{ICarMethods \{}
  \FunctionTok{move}\NormalTok{():}\DataTypeTok{void}\NormalTok{;}
\NormalTok{\}}
\end{Highlighting}
\end{Shaded}

Making the \texttt{Car} class implement the interfaces:

\begin{Shaded}
\begin{Highlighting}[numbers=left,,]
\KeywordTok{class} \NormalTok{Car }\KeywordTok{implements} \NormalTok{ICarProps, ICarMethods \{}
  \NormalTok{distance: number;}
  \FunctionTok{constructor} \NormalTok{() \{}
    \KeywordTok{this}\NormalTok{.}\FunctionTok{distance} \NormalTok{= }\DecValTok{5}\NormalTok{;}
  \NormalTok{\};}
  \FunctionTok{move}\NormalTok{():}\DataTypeTok{void} \NormalTok{\{}
    \KeywordTok{this}\NormalTok{.}\FunctionTok{distance} \NormalTok{+= }\DecValTok{1}\NormalTok{;}
  \NormalTok{\};}
\NormalTok{\}}
\end{Highlighting}
\end{Shaded}

The above example is silly, but it shows the point that a class can
implement one or more interfaces. Now if the class does not provide
implementations for any of the interfaces, the compiler will complain.
For example, if we leave out the \texttt{distance} instance variable,
the compiler will print out the following error:

\begin{quote}
error TS2420: Class `Car' incorrectly implements interface `ICarProps'.
Property `distance' is missing in type `Car'.
\end{quote}

\end{document}
