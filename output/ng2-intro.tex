\documentclass[12pt,]{article}
\usepackage{lmodern}
\usepackage{amssymb,amsmath}
\usepackage{ifxetex,ifluatex}
\usepackage{fixltx2e} % provides \textsubscript
\ifnum 0\ifxetex 1\fi\ifluatex 1\fi=0 % if pdftex
  \usepackage[T1]{fontenc}
  \usepackage[utf8]{inputenc}
\else % if luatex or xelatex
  \ifxetex
    \usepackage{mathspec}
    \usepackage{xltxtra,xunicode}
  \else
    \usepackage{fontspec}
  \fi
  \defaultfontfeatures{Mapping=tex-text,Scale=MatchLowercase}
  \newcommand{\euro}{€}
    \setmainfont{Palatino}
    \setsansfont{Century Gothic}
    \setmonofont[Mapping=tex-ansi]{Consolas}
\fi
% use upquote if available, for straight quotes in verbatim environments
\IfFileExists{upquote.sty}{\usepackage{upquote}}{}
% use microtype if available
\IfFileExists{microtype.sty}{%
\usepackage{microtype}
\UseMicrotypeSet[protrusion]{basicmath} % disable protrusion for tt fonts
}{}
\ifxetex
  \usepackage[setpagesize=false, % page size defined by xetex
              unicode=false, % unicode breaks when used with xetex
              xetex]{hyperref}
\else
  \usepackage[unicode=true]{hyperref}
\fi
\hypersetup{breaklinks=true,
            bookmarks=true,
            pdfauthor={Amin Meyghani},
            pdftitle={Introduction to Angular 2},
            colorlinks=true,
            citecolor=blue,
            urlcolor=blue,
            linkcolor=magenta,
            pdfborder={0 0 0}}
\urlstyle{same}  % don't use monospace font for urls
\usepackage{fancyhdr}
\pagestyle{fancy}
\pagenumbering{arabic}
\lhead{\itshape Introduction to Angular 2}
\chead{}
\rhead{\itshape{\nouppercase{\leftmark}}}
\lfoot{}
\cfoot{}
\rfoot{\thepage}
\usepackage{color}
\usepackage{fancyvrb}
\newcommand{\VerbBar}{|}
\newcommand{\VERB}{\Verb[commandchars=\\\{\}]}
\DefineVerbatimEnvironment{Highlighting}{Verbatim}{commandchars=\\\{\}}
% Add ',fontsize=\small' for more characters per line
\newenvironment{Shaded}{}{}
\newcommand{\KeywordTok}[1]{\textcolor[rgb]{0.00,0.00,1.00}{{#1}}}
\newcommand{\DataTypeTok}[1]{{#1}}
\newcommand{\DecValTok}[1]{{#1}}
\newcommand{\BaseNTok}[1]{{#1}}
\newcommand{\FloatTok}[1]{{#1}}
\newcommand{\ConstantTok}[1]{{#1}}
\newcommand{\CharTok}[1]{\textcolor[rgb]{0.00,0.50,0.50}{{#1}}}
\newcommand{\SpecialCharTok}[1]{\textcolor[rgb]{0.00,0.50,0.50}{{#1}}}
\newcommand{\StringTok}[1]{\textcolor[rgb]{0.00,0.50,0.50}{{#1}}}
\newcommand{\VerbatimStringTok}[1]{\textcolor[rgb]{0.00,0.50,0.50}{{#1}}}
\newcommand{\SpecialStringTok}[1]{\textcolor[rgb]{0.00,0.50,0.50}{{#1}}}
\newcommand{\ImportTok}[1]{{#1}}
\newcommand{\CommentTok}[1]{\textcolor[rgb]{0.00,0.50,0.00}{{#1}}}
\newcommand{\DocumentationTok}[1]{\textcolor[rgb]{0.00,0.50,0.00}{{#1}}}
\newcommand{\AnnotationTok}[1]{\textcolor[rgb]{0.00,0.50,0.00}{{#1}}}
\newcommand{\CommentVarTok}[1]{\textcolor[rgb]{0.00,0.50,0.00}{{#1}}}
\newcommand{\OtherTok}[1]{\textcolor[rgb]{1.00,0.25,0.00}{{#1}}}
\newcommand{\FunctionTok}[1]{{#1}}
\newcommand{\VariableTok}[1]{{#1}}
\newcommand{\ControlFlowTok}[1]{\textcolor[rgb]{0.00,0.00,1.00}{{#1}}}
\newcommand{\OperatorTok}[1]{{#1}}
\newcommand{\BuiltInTok}[1]{{#1}}
\newcommand{\ExtensionTok}[1]{{#1}}
\newcommand{\PreprocessorTok}[1]{\textcolor[rgb]{1.00,0.25,0.00}{{#1}}}
\newcommand{\AttributeTok}[1]{{#1}}
\newcommand{\RegionMarkerTok}[1]{{#1}}
\newcommand{\InformationTok}[1]{\textcolor[rgb]{0.00,0.50,0.00}{{#1}}}
\newcommand{\WarningTok}[1]{\textcolor[rgb]{0.00,0.50,0.00}{\textbf{{#1}}}}
\newcommand{\AlertTok}[1]{\textcolor[rgb]{1.00,0.00,0.00}{{#1}}}
\newcommand{\ErrorTok}[1]{\textcolor[rgb]{1.00,0.00,0.00}{\textbf{{#1}}}}
\newcommand{\NormalTok}[1]{{#1}}
\usepackage{graphicx,grffile}
\makeatletter
\def\maxwidth{\ifdim\Gin@nat@width>\linewidth\linewidth\else\Gin@nat@width\fi}
\def\maxheight{\ifdim\Gin@nat@height>\textheight\textheight\else\Gin@nat@height\fi}
\makeatother
% Scale images if necessary, so that they will not overflow the page
% margins by default, and it is still possible to overwrite the defaults
% using explicit options in \includegraphics[width, height, ...]{}
\setkeys{Gin}{width=\maxwidth,height=\maxheight,keepaspectratio}
\setlength{\parindent}{0pt}
\setlength{\parskip}{6pt plus 2pt minus 1pt}
\setlength{\emergencystretch}{3em}  % prevent overfull lines
\providecommand{\tightlist}{%
  \setlength{\itemsep}{0pt}\setlength{\parskip}{0pt}}
\setcounter{secnumdepth}{5}

\title{Introduction to Angular 2}
\author{Amin Meyghani}
\date{}

% Redefines (sub)paragraphs to behave more like sections
\ifx\paragraph\undefined\else
\let\oldparagraph\paragraph
\renewcommand{\paragraph}[1]{\oldparagraph{#1}\mbox{}}
\fi
\ifx\subparagraph\undefined\else
\let\oldsubparagraph\subparagraph
\renewcommand{\subparagraph}[1]{\oldsubparagraph{#1}\mbox{}}
\fi

\begin{document}
\maketitle

{
\hypersetup{linkcolor=black}
\setcounter{tocdepth}{5}
\tableofcontents
}
\section{Notes}\label{notes}

\begin{itemize}
\item
  The book assumes that you are working in a Unix-like environment. If
  you are on Windows you can use \href{https://www.cygwin.com/}{Cygwin}
  so that you can follow along with the bash terminal commands.
\item
  All the project files for the book are hosted on github:
  \url{https://github.com/st32lth/angular2-intro}. You can clone the
  repository and check out the project files. Throughout the book, you
  will see references to the project files. Those refer to this
  repository. For example,
  \texttt{angular2-intro/project-files/hello-angular} refers to the
  \texttt{hello-angular} folder inside the \texttt{project-files}
  folder.
\item
  Make sure you have \texttt{git} installed on your machine. That is,
  make sure you get an output for \texttt{git\ -\/-version}.
\item
  The book assumes that you have a working knowledge of JavaScript and
  Angular 1.x
\item
  Node is heavily used throughout the book. Make sure that you follow
  the ``Node'' chapter to install Node and set permissions correctly.
\item
  All the keyboard shortcuts are mac-based. But if you are using a
  non-mac machine, you can almost always replace \texttt{command} with
  \texttt{ctrl} and you should be good. For example, if you a see a
  shortcut like \texttt{command\ +\ shift\ +\ b}, you can use
  \texttt{ctrl\ +\ shift\ +\ b} where \texttt{ctrl} is obviously the
  \texttt{control} key.
\end{itemize}

\section{Installing Node}\label{installing-node}

You can use \href{https://github.com/creationix/nvm}{nvm} to install and
manage Node on your machine. Copy the install script and run it:

\begin{verbatim}
curl -o- https://raw.githubusercontent.com/creationix/nvm/v0.30.1/install.sh | bash
\end{verbatim}

After that, make a new terminal window and make sure that it is
installed, by running:

\begin{verbatim}
nvm --help
\end{verbatim}

Now you can use \texttt{nvm} to install Node \texttt{0.12.9} by running:

\begin{verbatim}
nvm install 0.12.9
\end{verbatim}

After that, nvm is going to load version 0.12.9 automatically. If it
doesn't, you can load it in the current shell, with:

\begin{verbatim}
nvm use 0.12.9
\end{verbatim}

Note that you can load any node version in the current shell with
\texttt{nvm\ use\ 0.x.y} after installing that version.

Also note that if you want to make \texttt{0.12.9} the default Node
version on your machine, you can do so by running the following:

\begin{verbatim}
nvm alias default 0.12.9
\end{verbatim}

Then you can verify that it is the default version by making a new
terminal window and typing \texttt{node\ -v}.

\subsection{Permissions}\label{permissions}

Never use \texttt{sudo} to install packages, never do
\texttt{sudo\ npm\ install\ \textless{}package\textgreater{}}. If you
get permission errors while installing without \texttt{sudo}, you can
own the folders instead. So for example, if you get an error like:

\begin{verbatim}
Error: EACCES, mkdir '/usr/local'
\end{verbatim}

you can own the folder with:

\begin{verbatim}
sudo chown -R `whoami` /usr/local
\end{verbatim}

You can own folders until Node doesn't complain.

\subsection{\texorpdfstring{Installing
\texttt{live-server}}{Installing live-server}}\label{installing-live-server}

Install a package to verify that node is installed and everything is
wired up correctly. We are going to use \texttt{live-server} through the
book. So let's install that with:

\begin{verbatim}
npm i -g live-server
\end{verbatim}

Then, you should be able to run \texttt{live-server} in any folder to
serve the content of that folder:

\begin{verbatim}
mdkir ~/Desktop/sample && cd $_
live-server .
\end{verbatim}

\section{Visual Studio Code}\label{visual-studio-code}

Visual Studio Code is a good IDE for developing web apps. In this
chapter we will look at installing and configuring VSCode.

\subsection{Visual Studio Code Basics}\label{visual-studio-code-basics}

\begin{itemize}
\item
  Install Visual Studio Code from: \url{https://code.visualstudio.com/}
\item
  You can open new projects by going to the
  \texttt{File\ \textgreater{}\ Open} tag, to etierh open a folder
  containing your project or a single file
\item
  Some useful keyboard shortcuts are:

  \begin{itemize}
  \tightlist
  \item
    \texttt{command\ +\ b}: to close/open the file navigator
  \item
    \texttt{command\ +\ shift\ +\ p}: to open the prompt
  \end{itemize}
\item
  To install extensions open the prompt with
  \texttt{command\ +\ shift\ +\ p} and type:

\begin{verbatim}
> install extension
\end{verbatim}
\item
  You can change the keyboard shortcuts settings from
  \texttt{Preferences\ \textgreater{}\ Keyboard\ Shortcuts}. Open the
  settings and then you can add your own shortcuts:

\begin{verbatim}
// Place your key bindings in this file to overwrite the defaults
[
  {
    "key": "cmd+t",
    "command": "workbench.action.quickOpen"
  },
  {
    "key": "shift+cmd+r",
    "command": "editor.action.format",
    "when": "editorTextFocus"
  }
]
\end{verbatim}
\end{itemize}

\subsection{Setting up VSCode for
TypeScript}\label{setting-up-vscode-for-typescript}

In this section we are going to set up Visual Studio Code for
TypeScript. The project files for this chapter are in
\href{https://github.com/st32lth/angular2-intro/tree/master/project-files/vscode}{\textbf{\texttt{angular2-intro/project-files/vscode}}}.
You can either follow along or check out the folder to see the final
result.

\subsubsection{Installing TypeScript}\label{installing-typescript}

Before anything, we need to install the TypeScript compiler. You can
install the TypeScript compiler with npm:

\begin{verbatim}
npm i typescript -g
\end{verbatim}

Then to verify that it is installed, run \texttt{tsc\ -v} to see the
version of the compiler. You will get an output like this:

\begin{verbatim}
message TS6029: Version 1.7.5
\end{verbatim}

In addition to the compiler, we also need to install the TypeScript
Definition manager for DefinitelyTyped (tsd). You can install tsd with:

\begin{verbatim}
npm i tsd -g
\end{verbatim}

Using TSD, you can search and install TypeScript definition files
directly from the community driven DefinitelyTyped repository. To verify
that tsd is installed, run tsd with the \texttt{version} flag:

\begin{verbatim}
tsd --version
\end{verbatim}

You should get an output like this:

\begin{verbatim}
>> tsd 0.6.5
\end{verbatim}

After \texttt{tsd} and \texttt{tsc} are installed, we can compile a
hello world program:

make a file called \texttt{hello.ts} on your desktop:

\begin{verbatim}
touch ~/Desktop/hello.ts
\end{verbatim}

Then, put some TypeScript code in the file:

\begin{verbatim}
echo "const adder = (a: number, b: number): number => a + b;" > ~/Desktop/hello.ts
\end{verbatim}

Then you can compile the file to JavaScript:

\begin{verbatim}
tsc ~/Desktop/hello.ts
\end{verbatim}

It should output a file in \texttt{Desktop/hello.js}:

\begin{Shaded}
\begin{Highlighting}[numbers=left,,]
\KeywordTok{var} \NormalTok{adder }\OperatorTok{=} \KeywordTok{function} \NormalTok{(a}\OperatorTok{,} \NormalTok{b) }\OperatorTok{\{} \ControlFlowTok{return} \NormalTok{a }\OperatorTok{+} \NormalTok{b}\OperatorTok{;} \OperatorTok{\};}
\end{Highlighting}
\end{Shaded}

Now that your TypeScript compiler setup, we can move on to configuring
Visual Studio Code.

\subsubsection{Add VSCode
Configurations}\label{add-vscode-configurations}

\begin{itemize}
\item
  First download and install Visual Studio Code from the VSCode
  \href{https://code.visualstudio.com/}{Website}
\item
  After installing VSCode, open it and then make a new window:
  \texttt{File\ \textgreater{}\ New\ Window}
\item
  Then, make a folder on your desktop for a new project:
  \texttt{mkdir\ \textasciitilde{}/Desktop/vscode-demo}
\item
  After that, open the folder in VSCode:
  \texttt{File\ \textgreater{}\ open} and select the
  \texttt{vscode-demo} folder on your desktop.
\item
  Now we need to make three configuration files:

  \begin{enumerate}
  \def\labelenumi{\arabic{enumi}.}
  \tightlist
  \item
    \href{http://json.schemastore.org/tsconfig}{\texttt{tsconfig.json}}:
    configuration for the TypeScript compiler
  \item
    \texttt{tasks.json}: Task configuration for VSCode to watch and
    compile files
  \item
    \texttt{launch.json}: Configuration for the debugger
  \end{enumerate}
\end{itemize}

The \texttt{tsconfig.json} file should be in the root of the project.
Let's make the file and put the following in it:

\begin{Shaded}
\begin{Highlighting}[numbers=left,,]
\OperatorTok{\{}
  \StringTok{"compilerOptions"}\OperatorTok{:} \OperatorTok{\{}
    \StringTok{"experimentalDecorators"}\OperatorTok{:} \KeywordTok{true}\OperatorTok{,}
    \StringTok{"emitDecoratorMetadata"}\OperatorTok{:} \KeywordTok{true}\OperatorTok{,}
    \StringTok{"module"}\OperatorTok{:} \StringTok{"commonjs"}\OperatorTok{,}
    \StringTok{"target"}\OperatorTok{:} \StringTok{"es5"}\OperatorTok{,}
    \StringTok{"sourceMap"}\OperatorTok{:} \KeywordTok{true}\OperatorTok{,}
    \StringTok{"outDir"}\OperatorTok{:} \StringTok{"output"}\OperatorTok{,}
    \StringTok{"watch"}\OperatorTok{:} \KeywordTok{true}
  \OperatorTok{\}}
\OperatorTok{\}}
\end{Highlighting}
\end{Shaded}

Now to make the \texttt{tasks.json} file. Open the prompt with
\texttt{command\ +\ shift\ +\ p} and type:

\begin{verbatim}
> configure task runner
\end{verbatim}

Then put the following in the file and save the file:

\begin{Shaded}
\begin{Highlighting}[numbers=left,,]
\OperatorTok{\{}
  \StringTok{"version"}\OperatorTok{:} \StringTok{"0.1.0"}\OperatorTok{,}
  \StringTok{"command"}\OperatorTok{:} \StringTok{"tsc"}\OperatorTok{,}
  \StringTok{"showOutput"}\OperatorTok{:} \StringTok{"silent"}\OperatorTok{,}
  \StringTok{"isShellCommand"}\OperatorTok{:} \KeywordTok{true}\OperatorTok{,}
  \StringTok{"problemMatcher"}\OperatorTok{:} \StringTok{"$tsc"}
\OperatorTok{\}}
\end{Highlighting}
\end{Shaded}

The last thing that we need to set up is the debugger, i.e.~the
\texttt{launch.json} file. Right click on the \texttt{.vscode} folder in
the file navigator and make a new file called \texttt{launch.json} and
put in the following:

\begin{Shaded}
\begin{Highlighting}[numbers=left,,]
\OperatorTok{\{}
  \StringTok{"version"}\OperatorTok{:} \StringTok{"0.1.0"}\OperatorTok{,}
  \StringTok{"configurations"}\OperatorTok{:} \NormalTok{[}
    \OperatorTok{\{}
      \StringTok{"name"}\OperatorTok{:} \StringTok{"TS Debugger"}\OperatorTok{,}
      \StringTok{"type"}\OperatorTok{:} \StringTok{"node"}\OperatorTok{,}
      \StringTok{"program"}\OperatorTok{:} \StringTok{"main.ts"}\OperatorTok{,}
      \StringTok{"stopOnEntry"}\OperatorTok{:} \KeywordTok{false}\OperatorTok{,}
      \StringTok{"sourceMaps"}\OperatorTok{:} \KeywordTok{true}\OperatorTok{,}
      \StringTok{"outDir"}\OperatorTok{:} \StringTok{"output"}
    \OperatorTok{\}}
  \NormalTok{]}
\OperatorTok{\}}
\end{Highlighting}
\end{Shaded}

After you save the file, you should be able to see the debugger in the
debugger dropdown options.

Now, we are ready to make the \texttt{main.ts} file in the root of the
project:

\textbf{\texttt{main.ts}}

\begin{Shaded}
\begin{Highlighting}[numbers=left,,]
\DataTypeTok{const} \NormalTok{sum = (a: number, b: number): number => a + b;}
\DataTypeTok{const} \NormalTok{r = }\FunctionTok{sum}\NormalTok{(}\DecValTok{1}\NormalTok{,}\DecValTok{2}\NormalTok{);}
\NormalTok{console.}\FunctionTok{log}\NormalTok{(r);}
\end{Highlighting}
\end{Shaded}

Now you can start the task to watch the files and compile as you work.
Open the prompt with \texttt{command\ +\ shift\ +\ p} and type:

\begin{verbatim}
> run build tasks
\end{verbatim}

you can also use the \texttt{command\ +\ shift\ +\ b} keyboard shortcut
instead. This will start the debugger and watch the files. After making
a change to \texttt{main.ts}, you should be able to see the output in
the \texttt{output} folder.

After the build task is running, we can put a breakpoint anywhere in our
TypeScript code. Let's put a breakpoint on line 2 by clicking on the
margin. Then start the debugger by going to the debugger tab and
clicking the green play icon.

Now you should see that the program will stop at the breakpoint and you
should be able to step over or into your program.

To stop the task you can terminate it. Open the prompt and type:

\begin{verbatim}
> terminate running task
\end{verbatim}

You can learn more about running TypeScript with VSCode on MSDN's
\href{http://blogs.msdn.com/b/typescript/archive/2015/04/30/using-typescript-in-visual-studio-code.aspx}{blog}.

\subsection{Running VSCode from the
Terminal}\label{running-vscode-from-the-terminal}

If you want to run VSCode from the terminal, you can follow the
\href{https://code.visualstudio.com/Docs/editor/setup}{guide} on
VSCode's website. Below is the summary of the guide:

\textbf{MAC}

Add the following to your ``bash'' file:

\begin{verbatim}
function code () { VSCODE_CWD="$PWD" open -n -b "com.microsoft.VSCode" --args $*; }
\end{verbatim}

\textbf{Linux}

\begin{verbatim}
sudo ln -s /path/to/vscode/Code /usr/local/bin/code
\end{verbatim}

\textbf{Windows}

You might need to log off after the installation for the change to the
PATH environmental variable to take effect.

\hypertarget{debugging-app-from-vscode}{\subsection{Debugging App from
VSCode}\label{debugging-app-from-vscode}}

The ``vscode-chrome-debug'' extension allows you to attach VSCode to a
running instance of chrome. This makes it very convenient because you
can put breakpoints in your TypeScript code and run the debugger to
debug your app. Let's get started.

In order to install the
\href{https://github.com/Microsoft/vscode-chrome-debug}{extension} open
the prompt in VSCode with \texttt{command\ +\ shift\ +\ p} and type:

\begin{verbatim}
> install extension
\end{verbatim}

hit enter and then type:

\begin{verbatim}
debugger for chrome
\end{verbatim}

Then just click on the result to install the extension. Restart VSCode
when you are prompted.

After installing the extension, we need to update or create a
\texttt{launch.json} file for debugging. You can create one in the
\texttt{.vscode} folder. After you created the file, put in the
following:

\begin{verbatim}
{
  "version": "0.1.0",
  "configurations": [
    {
      "name": "Launch Chrome Debugger",
      "type": "chrome",
      "request": "launch",
      "url": "http://localhost:8080",
      "sourceMaps": true,
      "webRoot": ".",
      "runtimeExecutable": "/Applications/Google Chrome.app/Contents/MacOS/Google Chrome",
      "runtimeArgs": ["--remote-debugging-port=9222", "--incognito"]
    }
  ]
}
\end{verbatim}

\textbf{Notes:}

\begin{itemize}
\item
  Depending on your platform you need to change the
  \texttt{runtimeExecutable} path to Chrome's executable path. After
  configuring the debugger you need to have a server running serving the
  app. You can change the \texttt{url} value accordingly. Also make sure
  that the \texttt{webRoot} path is set to the root of your web server.
\item
  After that it is a good idea to close all the instances of chrome.
  Then, put a breakpoint in your code and run the debugger. If
  everything is set up correctly, you should see an instance of chrome
  running in incognito mode. To trigger the breakpoint, just reload the
  page and you should be able to see the debugger paused at the
  breakpoint.
\item
  Also make sure that you have the compiler running so that you can use
  the JavaScript output and the sourcemaps to use the debugger. See the
  TypeScript and VSCode set up for more details.
\end{itemize}

\section{TypeScript Crash-course}\label{typescript-crash-course}

In this chapter we will quickly go through the most important concepts
in TypeScript so that you can have a better understanding of Angular
code that you will write. Knowing TypeScript definitely helps to
understand Angular, but again it is not a requirement. The project files
for this chapter are in
\href{https://github.com/st32lth/angular2-intro/tree/master/project-files/typescript}{\textbf{\texttt{angular2-intro/project-files/typescript}}}.

\subsection{TypeScript Basics}\label{typescript-basics}

\begin{itemize}
\item
  TypeScript is a superset of JavaScript with additional features, among
  which optional types is the most notable. This means that any valid
  JavaScript code (ES 2015/2016\ldots{}) is valid TypeScript code. You
  can basically change the extension of the file to \texttt{.ts} and
  compile it with the the TypeScript compiler.
\item
  TypeScript defines 7 primary types:

  \begin{itemize}
  \tightlist
  \item
    boolean: \texttt{var\ isDone:\ boolean\ =\ false;}
  \item
    number: \texttt{var\ height:\ number\ =\ 6;}
  \item
    string: \texttt{var\ name:\ string\ =\ "bob";}
  \item
    array: \texttt{var\ list:number{[}{]}\ =\ {[}1,\ 2,\ 3{]};} also
    \texttt{var\ list:Array\textless{}number\textgreater{}\ =\ {[}1,\ 2,\ 3{]};}
  \item
    enum: \texttt{enum\ Color\ \{Red,\ Green,\ Blue\};}
  \item
    any: \texttt{var\ notSure:\ any\ =\ 4;}
  \item
    void:
    \texttt{function\ hello():\ void\ \{\ console.log(\textquotesingle{}hello\textquotesingle{});\ \}}
  \end{itemize}
\end{itemize}

\subsection{Interface}\label{interface}

\begin{itemize}
\tightlist
\item
  An Interface is defined using the \texttt{interface} keyword
\item
  Interfaces are used only during compilation time to check types
\item
  By convention, interface definitions start with an \texttt{I}, e.g. :
  \texttt{IPoint}
\item
  Interfaces are used in classical object oriented programming as a
  design tool
\item
  Interfaces don't contain implementations
\item
  They provide definitions only
\item
  When an object implements an interface, it must adhere to the contract
  defined by the interface
\item
  An interface defines what properties and methods an object must
  implement
\item
  If an object implements an interface, it must adhere to the contract.
  If it doesn't the compiler will let us know.
\item
  Interfaces also define custom types
\end{itemize}

\subsubsection{Basic Interface}\label{basic-interface}

Below is an example of an Interface that defines two properties and
three methods that implementers should provide implementations for:

\begin{Shaded}
\begin{Highlighting}[numbers=left,,]
\KeywordTok{interface} \NormalTok{IMyInterface \{}
  \CommentTok{// some properties}
  \NormalTok{id: number;}
  \NormalTok{name: string;}

  \CommentTok{// some methods}
  \FunctionTok{method}\NormalTok{(): }\DataTypeTok{void}\NormalTok{;}
  \FunctionTok{methodWithReturnVal}\NormalTok{():number;}
  \FunctionTok{sum}\NormalTok{(nums: number[]):number;}
\NormalTok{\}}
\end{Highlighting}
\end{Shaded}

Using the interface above we can create an object that adheres to the
interface:

\begin{Shaded}
\begin{Highlighting}[numbers=left,,]
\NormalTok{let myObj: IMyInterface = \{}
  \NormalTok{id: }\DecValTok{2}\NormalTok{,}
  \NormalTok{name: 'some name',}

  \FunctionTok{method}\NormalTok{() \{ console.}\FunctionTok{log}\NormalTok{('hello'); \},}
  \FunctionTok{methodWithReturnVal} \NormalTok{() \{ }\KeywordTok{return} \DecValTok{2}\NormalTok{; \},}
  \FunctionTok{sum}\NormalTok{(numbers) \{}
    \KeywordTok{return} \NormalTok{numbers.}\FunctionTok{reduce}\NormalTok{( (a,b) => \{ }\KeywordTok{return} \NormalTok{a + b \} );}
  \NormalTok{\}}
\NormalTok{\};}
\end{Highlighting}
\end{Shaded}

Notice that we had to provide values to \textbf{all} the properties
defined by the Interface, and the implementations for \textbf{all} the
methods defined by the Interface.

And then of course you can use your object methods to perform
operations:

\begin{Shaded}
\begin{Highlighting}[numbers=left,,]
\NormalTok{let sum = myObj.}\FunctionTok{sum}\NormalTok{([}\DecValTok{1}\NormalTok{,}\DecValTok{2}\NormalTok{,}\DecValTok{3}\NormalTok{,}\DecValTok{4}\NormalTok{,}\DecValTok{5}\NormalTok{]); }\CommentTok{// -> 15}
\end{Highlighting}
\end{Shaded}

\subsubsection{Classes as Interfaces}\label{classes-as-interfaces}

Because classes define types as well, they can also be used as
interfaces. If you have an interface you can extend it with a class for
example:

\begin{Shaded}
\begin{Highlighting}[numbers=left,,]
\KeywordTok{class} \NormalTok{Point \{}
  \NormalTok{x: number;}
  \NormalTok{y: number;}
\NormalTok{\}}
\KeywordTok{interface} \NormalTok{Point3d }\KeywordTok{extends} \NormalTok{Point \{}
  \NormalTok{z: number;}
\NormalTok{\}}
\DataTypeTok{const} \NormalTok{point3d: Point3d = \{x: }\DecValTok{1}\NormalTok{, y: }\DecValTok{2}\NormalTok{, z: }\DecValTok{3}\NormalTok{\};}
\NormalTok{console.}\FunctionTok{log}\NormalTok{(point3d.}\FunctionTok{x}\NormalTok{); }\CommentTok{// -> 1}
\end{Highlighting}
\end{Shaded}

First we are defining a class called \texttt{Point} that defines two
fields. Then we define an interface called \texttt{Point3d} that extends
the \texttt{Point} by adding a third field. An then we create a point of
type \texttt{point3d} and assign a value to it. We read the value and it
outputs \texttt{1}.

\subsection{Classes}\label{classes}

\begin{itemize}
\tightlist
\item
  Classes are heavily used in classical object oriented programming
\item
  It defines what an object is and what it can do
\item
  A class is defined using the \texttt{class} keyword followed by a name
\item
  By convention, the name of the class start with an uppercase letter
\item
  A class can be used to create multiple objects (instances) of the same
  class
\item
  An object is created from a class using the \texttt{new} keyword
\item
  A class can have a \texttt{constructor} which is called when an object
  is made from the class
\item
  Properties of a class are called instance variables and its functions
  are called the class methods
\item
  Access modifiers can be used to make them public or private
\item
  The instance variables are attached to the instance itself but not the
  prototype
\item
  Methods however are attached to the prototype object as opposed to the
  instance itself
\item
  Classes can inherit functionality from other classes, but you should
  \href{https://medium.com/javascript-scene/the-two-pillars-of-javascript-ee6f3281e7f3\#.oc5pdevwh}{favor
  composition over inheritance} or make sure you know
  \href{https://medium.com/@dtinth/es6-class-classical-inheritance-20f4726f4c4\#.xdif2m42e}{when
  to use it}
\item
  Classes can implement interfaces
\end{itemize}

Let's make a class definition for a car and incrementally add more
things to it. The project files for this section are in
\href{https://github.com/st32lth/angular2-intro/tree/master/project-files/typescript/classes/basic-class}{\textbf{\texttt{angular2-intro/project-files/typescript/classes/basic-class}}}.

\subsubsection{Adding an Instance
Variable}\label{adding-an-instance-variable}

The \texttt{Car} class definition can be very simple and can define only
a single instance variable that all cars can have:

\begin{Shaded}
\begin{Highlighting}[numbers=left,,]
\KeywordTok{class} \NormalTok{Car \{}
  \NormalTok{distance: number;}
\NormalTok{\}}
\end{Highlighting}
\end{Shaded}

\begin{itemize}
\tightlist
\item
  \texttt{Car} is the name of the class, which also defines the custom
  type \texttt{Car}
\item
  \texttt{distance} is a property that tracks the distance that car has
  traveled
\item
  Distance is of type \texttt{number} and only accepts \texttt{number}
  type.
\end{itemize}

Now that we have the definition for a car, we can create a car from the
definition:

\begin{Shaded}
\begin{Highlighting}[numbers=left,,]
\NormalTok{let myCar:Car = }\KeywordTok{new} \FunctionTok{Car}\NormalTok{();}
\NormalTok{myCar.}\FunctionTok{distance} \NormalTok{= }\DecValTok{0}\NormalTok{;}
\end{Highlighting}
\end{Shaded}

\begin{itemize}
\tightlist
\item
  \texttt{myCar:Car} means that \texttt{myCar} is of type \texttt{Car}
\item
  \texttt{new\ Car()} creates an instance from the \texttt{Car}
  definition.
\item
  \texttt{myCar.distance\ =\ 0} sets the initial value of the
  \texttt{distance} to 0 for the newly created \texttt{car}
\end{itemize}

\subsubsection{Adding a Method}\label{adding-a-method}

So far our car doesn't have any definitions for any actions. Let's
define a \texttt{move} method that all the cars can have:

\begin{Shaded}
\begin{Highlighting}[numbers=left,,]
\KeywordTok{class} \NormalTok{Car \{}
  \NormalTok{distance: number;}
  \FunctionTok{move}\NormalTok{():}\DataTypeTok{void} \NormalTok{\{}
    \KeywordTok{this}\NormalTok{.}\FunctionTok{distance} \NormalTok{+= }\DecValTok{1}\NormalTok{;}
  \NormalTok{\}}
\NormalTok{\}}
\end{Highlighting}
\end{Shaded}

\begin{itemize}
\tightlist
\item
  \texttt{move():void} means that \texttt{move} is a method that does
  not return any value, hence \texttt{void}.
\item
  The body of the method is defined in \texttt{\{\ \}}
\item
  \texttt{this} refers to the instance, therefore \texttt{this.distance}
  points to the \texttt{distance} property defined on the car instance.
\item
  Now you can call the \texttt{move} method on the car instance to
  increment the \texttt{distance} value by 1:
\end{itemize}

\begin{Shaded}
\begin{Highlighting}[numbers=left,,]
\NormalTok{myCar.}\FunctionTok{move}\NormalTok{();}
\NormalTok{console.}\FunctionTok{log}\NormalTok{(myCar.}\FunctionTok{distance}\NormalTok{) }\CommentTok{// -> 1}
\end{Highlighting}
\end{Shaded}

\subsubsection{Using Access Modifiers}\label{using-access-modifiers}

If you wanted to tell the compiler that the \texttt{distance} variable
is private and can only be used by the object itself, you can use the
\texttt{private} modifier before the name of the property:

\begin{Shaded}
\begin{Highlighting}[numbers=left,,]
\KeywordTok{class} \NormalTok{Car \{}
  \KeywordTok{private} \NormalTok{distance: number;}
  \FunctionTok{constructor} \NormalTok{() \{}
    \NormalTok{...}
  \NormalTok{\}}
  \NormalTok{...}
\NormalTok{\}}
\end{Highlighting}
\end{Shaded}

\begin{itemize}
\item
  There are 3 main access modifiers in TypeScript: \texttt{private},
  \texttt{public}, and \texttt{protected}:
\item
  \texttt{private} modifier means that the property or the method is
  only defined inside the class only.
\item
  \texttt{protected} modifier means that the property or the method is
  only accessible inside the class and the classes derived from the
  class.
\item
  \texttt{public} is the default modifier which means the property or
  the method is the accessible everywhere and can be accessed by anyone.
\end{itemize}

\subsubsection{Adding a constructor}\label{adding-a-constructor}

A \texttt{constructor} is a special method that gets called when an
instance is created from a class. A class may contain at most one
constructor declaration. If a class contains no constructor declaration,
an automatic constructor is provided.

Let's add a constructor to the \texttt{Car} class that initializes the
\texttt{distance} value to 0. This means that all the cars that are
crated from this class, will have their \texttt{distance} set to 0
automatically:

\begin{Shaded}
\begin{Highlighting}[numbers=left,,]
\KeywordTok{class} \NormalTok{Car \{}
  \NormalTok{distance: number;}
  \FunctionTok{constructor} \NormalTok{() \{}
    \KeywordTok{this}\NormalTok{.}\FunctionTok{distance} \NormalTok{= }\DecValTok{0}\NormalTok{;}
  \NormalTok{\}}
  \FunctionTok{move}\NormalTok{():}\DataTypeTok{void} \NormalTok{\{}
    \KeywordTok{this}\NormalTok{.}\FunctionTok{distance} \NormalTok{+= }\DecValTok{1}\NormalTok{;}
  \NormalTok{\}}
\NormalTok{\}}
\end{Highlighting}
\end{Shaded}

\begin{itemize}
\tightlist
\item
  \texttt{constructor()} is called automatically when a new car is
  created
\item
  Parameters are passed to the constructor in the \texttt{()}
\item
  The body of the constructor is defined in the \texttt{\{\ \}}
\end{itemize}

Now, let's customize the car's constructor to accept \texttt{distance}
as a parameter:

\begin{Shaded}
\begin{Highlighting}[numbers=left,,]
\KeywordTok{class} \NormalTok{Car \{}
  \KeywordTok{private} \NormalTok{distance: number;}
  \FunctionTok{constructor} \NormalTok{(distance) \{}
    \KeywordTok{this}\NormalTok{.}\FunctionTok{distance} \NormalTok{= distance;}
  \NormalTok{\}}
\NormalTok{\}}
\end{Highlighting}
\end{Shaded}

\begin{itemize}
\tightlist
\item
  On line 3 we are passing distance as a parameter. This means that when
  a new instance is created, a value should be passed in to set the
  distance of the car.
\item
  On line 4 we are assigning the value of distance to the value that is
  passed in
\end{itemize}

This pattern is so common that TypeScript has a shorthand for it:

\begin{Shaded}
\begin{Highlighting}[numbers=left,,]
\KeywordTok{class} \NormalTok{Car \{}
  \FunctionTok{constructor} \NormalTok{(}\KeywordTok{private} \NormalTok{distance) \{}
  \NormalTok{\}}
\NormalTok{\}}
\end{Highlighting}
\end{Shaded}

Note that the only thing that we had to do was to add
\texttt{private\ distance} in the constructor parameter and remove the
\texttt{this.distance} and \texttt{distance:\ number}. TypeScript will
automatically generate that. Below is the JavaScript outputed by
TypeScript:

\begin{Shaded}
\begin{Highlighting}[numbers=left,,]
\KeywordTok{var} \NormalTok{Car }\OperatorTok{=} \NormalTok{(}\KeywordTok{function} \NormalTok{() }\OperatorTok{\{}
  \KeywordTok{function} \AttributeTok{Car}\NormalTok{(distance) }\OperatorTok{\{}
    \KeywordTok{this}\NormalTok{.}\AttributeTok{distance} \OperatorTok{=} \NormalTok{distance}\OperatorTok{;}
  \OperatorTok{\}}
  \ControlFlowTok{return} \NormalTok{Car}\OperatorTok{;}
\OperatorTok{\}}\NormalTok{)()}\OperatorTok{;}
\end{Highlighting}
\end{Shaded}

Now that our car expects a \texttt{distance} we have to always supply a
value for the distance when creating a car. You can define default
values if you want so that the car is instantiated with a default value
for the distance if none is given:

\begin{Shaded}
\begin{Highlighting}[numbers=left,,]
\KeywordTok{class} \NormalTok{Car \{}
  \FunctionTok{constructor} \NormalTok{(}\KeywordTok{private} \NormalTok{distance = }\DecValTok{0}\NormalTok{) \{}
  \NormalTok{\}}
  \FunctionTok{getDistance}\NormalTok{():number \{ }\KeywordTok{return} \KeywordTok{this}\NormalTok{.}\FunctionTok{distance}\NormalTok{; \}}
\NormalTok{\}}
\end{Highlighting}
\end{Shaded}

Now if I forget to pass a value for the \texttt{distance}, it is going
to be set to zero by default:

\begin{Shaded}
\begin{Highlighting}[numbers=left,,]
\DataTypeTok{const} \NormalTok{mycar = }\KeywordTok{new} \FunctionTok{Car}\NormalTok{();}
\NormalTok{console.}\FunctionTok{log}\NormalTok{(mycar.}\FunctionTok{getDistance}\NormalTok{()); }\CommentTok{//-> 0}
\end{Highlighting}
\end{Shaded}

Note that if you pass a value, it will override the default value:

\begin{Shaded}
\begin{Highlighting}[numbers=left,,]
\DataTypeTok{const} \NormalTok{mycar = }\KeywordTok{new} \FunctionTok{Car}\NormalTok{(}\DecValTok{5}\NormalTok{);}
\NormalTok{console.}\FunctionTok{log}\NormalTok{(mycar.}\FunctionTok{getDistance}\NormalTok{()); }\CommentTok{//-> 5}
\end{Highlighting}
\end{Shaded}

\subsubsection{Setters and Getters
(Accessors)}\label{setters-and-getters-accessors}

It is a very common pattern to have setters and getters for properties
of a class. TypeScript provides a very simple syntax to achieve that.
Let's take our example above and add a setter and getter for the
distance property. But before that we are going to rename
\texttt{distance} to \texttt{\_distance} to make it explicit that it is
private. It is not required but it is a common pattern to prefix private
properties with an underscore.

\begin{Shaded}
\begin{Highlighting}[numbers=left,,]
\KeywordTok{class} \NormalTok{Car \{}
  \FunctionTok{constructor} \NormalTok{(}\KeywordTok{private} \NormalTok{_distance = }\DecValTok{0}\NormalTok{) \{\}}
  \FunctionTok{getDistance}\NormalTok{():number \{ }\KeywordTok{return} \KeywordTok{this}\NormalTok{.}\FunctionTok{_distance}\NormalTok{; \}}
\NormalTok{\}}
\end{Highlighting}
\end{Shaded}

In order to create the getter method, we are going to use the
\texttt{get} keyword and the name for the property followed by
\texttt{()}:

\begin{Shaded}
\begin{Highlighting}[numbers=left,,]
\KeywordTok{class} \NormalTok{Car \{}
  \FunctionTok{constructor} \NormalTok{(}\KeywordTok{private} \NormalTok{_distance = }\DecValTok{0}\NormalTok{) \{\}}
  \NormalTok{get }\FunctionTok{distance}\NormalTok{() \{ }\KeywordTok{return} \KeywordTok{this}\NormalTok{.}\FunctionTok{_distance}\NormalTok{; \}}
\NormalTok{\}}
\end{Highlighting}
\end{Shaded}

Now we can get the value of \texttt{distance}:

\begin{Shaded}
\begin{Highlighting}[numbers=left,,]
\DataTypeTok{const} \NormalTok{car2 = }\KeywordTok{new} \FunctionTok{Car}\NormalTok{(}\DecValTok{5}\NormalTok{);}
\NormalTok{console.}\FunctionTok{log}\NormalTok{(car2.}\FunctionTok{distance}\NormalTok{) }\CommentTok{//-> 5}
\end{Highlighting}
\end{Shaded}

Note on line 2 that we didn't call a function. Behind the scenes,
TypeScript creates a property for us, that's why it is not a method.
Below is the relevant generated JavaScript:

\begin{Shaded}
\begin{Highlighting}[numbers=left,,]
\VariableTok{Object}\NormalTok{.}\AttributeTok{defineProperty}\NormalTok{(}\VariableTok{Car}\NormalTok{.}\AttributeTok{prototype}\OperatorTok{,} \StringTok{"distance"}\OperatorTok{,} \OperatorTok{\{}
  \DataTypeTok{get}\OperatorTok{:} \KeywordTok{function} \NormalTok{() }\OperatorTok{\{} \ControlFlowTok{return} \KeywordTok{this}\NormalTok{.}\AttributeTok{_distance}\OperatorTok{;} \OperatorTok{\},}
  \DataTypeTok{enumerable}\OperatorTok{:} \KeywordTok{true}\OperatorTok{,}
  \DataTypeTok{configurable}\OperatorTok{:} \KeywordTok{true}
\OperatorTok{\}}\NormalTok{)}\OperatorTok{;}
\end{Highlighting}
\end{Shaded}

JavaScript behind the scenes calls the get function for you to get the
value, and that's why we simply did \texttt{car2.distance} as opposed to
\texttt{car2.distance()}. For more information about
\texttt{Object.defineProperty} checkout the
\href{https://developer.mozilla.org/en-US/docs/Web/JavaScript/Reference/Global_Objects/Object/defineProperty}{MDN}
docs.

Similar to the getter, we can define a setter as well:

\begin{Shaded}
\begin{Highlighting}[numbers=left,,]
\KeywordTok{class} \NormalTok{Car \{}
  \FunctionTok{constructor} \NormalTok{(}\KeywordTok{private} \NormalTok{_distance = }\DecValTok{0}\NormalTok{) \{\}}
  \NormalTok{get }\FunctionTok{distance}\NormalTok{() \{ }\KeywordTok{return} \KeywordTok{this}\NormalTok{.}\FunctionTok{_distance}\NormalTok{; \}}
  \NormalTok{set }\FunctionTok{distance}\NormalTok{(newDistance: number) \{ }\KeywordTok{this}\NormalTok{.}\FunctionTok{_distance} \NormalTok{= newDistance; \}}
\NormalTok{\}}
\end{Highlighting}
\end{Shaded}

Now we can both get and set the distance value:

\begin{Shaded}
\begin{Highlighting}[numbers=left,,]
\DataTypeTok{const} \NormalTok{coolCar = }\KeywordTok{new} \FunctionTok{Car}\NormalTok{();}
\NormalTok{console.}\FunctionTok{log}\NormalTok{(coolCar.}\FunctionTok{distance}\NormalTok{); }\CommentTok{// -> 0}

\NormalTok{coolCar.}\FunctionTok{distance} \NormalTok{= }\DecValTok{55}\NormalTok{;}
\NormalTok{console.}\FunctionTok{log}\NormalTok{(coolCar.}\FunctionTok{distance}\NormalTok{); }\CommentTok{// -> 55}
\end{Highlighting}
\end{Shaded}

Note that if we take out the setter, we won't be able to assign a new
value to \texttt{distance}.

\subsubsection{Static Methods and
Properties}\label{static-methods-and-properties}

Static methods and properties belong to the class but not the instances.
For example, the \texttt{Array.isArray} method is only accessible
through the \texttt{Array} but not an instance of an array:

\begin{Shaded}
\begin{Highlighting}[numbers=left,,]
\KeywordTok{var} \NormalTok{x }\OperatorTok{=} \NormalTok{[]}\OperatorTok{;}
\VariableTok{x}\NormalTok{.}\AttributeTok{isArray} \CommentTok{// -> undefined}
\VariableTok{Array}\NormalTok{.}\AttributeTok{isArray}\NormalTok{(x) }\CommentTok{//-> true}
\end{Highlighting}
\end{Shaded}

\begin{itemize}
\tightlist
\item
  On line 2 we are trying to access the \texttt{isArray} method, but
  obviously it is not defined because \texttt{isArray} is a static
  method.
\item
  On line three we are calling the static \texttt{isArray} method from
  \texttt{Array} and we can check if \texttt{x} is an array.
\end{itemize}

If you look at the
\href{https://developer.mozilla.org/en-US/docs/Web/JavaScript/Reference/Global_Objects/Array/isArray}{Array}
documentation you can see that methods and properties are either defined
on the \texttt{Array.prototype} or \texttt{Array}:

\begin{itemize}
\tightlist
\item
  \texttt{Array.prototype.x}: makes \texttt{x} available to all the
  instances of \texttt{Array}
\item
  \texttt{Array.x}: \texttt{x} is static and only available through
  \texttt{Array}.
\end{itemize}

Now that we have some context, let's see how you can define static
methods and properties in TypeScript. Consider the code below:

\begin{Shaded}
\begin{Highlighting}[numbers=left,,]
\KeywordTok{class} \NormalTok{Car \{}
  \DataTypeTok{static} \NormalTok{controls: \{isAuto: }\DataTypeTok{boolean} \NormalTok{\} = \{}
    \NormalTok{isAuto: }\KeywordTok{true}
  \NormalTok{\};}
  \DataTypeTok{static} \FunctionTok{isAuto}\NormalTok{():}\DataTypeTok{boolean} \NormalTok{\{}
    \KeywordTok{return} \NormalTok{Car.}\FunctionTok{controls}\NormalTok{.}\FunctionTok{isAuto}\NormalTok{;}
  \NormalTok{\}}
  \FunctionTok{constructor} \NormalTok{(}\KeywordTok{private} \NormalTok{_distance = }\DecValTok{0}\NormalTok{) \{\}}
  \NormalTok{get }\FunctionTok{distance}\NormalTok{() \{ }\KeywordTok{return} \KeywordTok{this}\NormalTok{.}\FunctionTok{_distance}\NormalTok{; \}}
\NormalTok{\}}

\NormalTok{console.}\FunctionTok{log}\NormalTok{(Car.}\FunctionTok{controls}\NormalTok{); }\CommentTok{// -> \{ isAuto: true \}}
\NormalTok{console.}\FunctionTok{log}\NormalTok{(Car.}\FunctionTok{isAuto}\NormalTok{()); }\CommentTok{// -> true}
\end{Highlighting}
\end{Shaded}

\begin{itemize}
\tightlist
\item
  On line 2 we are defining a static property called \texttt{controls}
  using the \texttt{static} modifier. Then we specify the form and then
  assign a value for it.
\item
  On line 5 we are defining a static method called \texttt{isAuto} using
  the the \texttt{static} modifier. This method simply returns the value
  of \texttt{isAuto} from the static \texttt{control} object. Not that
  we get access to the class using the name of the class as opposed to
  using \texttt{this}. i.e. \texttt{return\ Car.controls.isAuto}
\end{itemize}

\subsubsection{Implementing an
Interface}\label{implementing-an-interface}

Classes can implement one or multiple interfaces. We can make the
\texttt{Car} class implement two interfaces:

\begin{Shaded}
\begin{Highlighting}[numbers=left,,]
\KeywordTok{interface} \NormalTok{ICarProps \{}
  \NormalTok{distance: number;}
\NormalTok{\}}
\KeywordTok{interface} \NormalTok{ICarMethods \{}
  \FunctionTok{move}\NormalTok{():}\DataTypeTok{void}\NormalTok{;}
\NormalTok{\}}
\end{Highlighting}
\end{Shaded}

Making the \texttt{Car} class implement the interfaces:

\begin{Shaded}
\begin{Highlighting}[numbers=left,,]
\KeywordTok{class} \NormalTok{Car }\KeywordTok{implements} \NormalTok{ICarProps, ICarMethods \{}
  \NormalTok{distance: number;}
  \FunctionTok{constructor} \NormalTok{() \{}
    \KeywordTok{this}\NormalTok{.}\FunctionTok{distance} \NormalTok{= }\DecValTok{5}\NormalTok{;}
  \NormalTok{\};}
  \FunctionTok{move}\NormalTok{():}\DataTypeTok{void} \NormalTok{\{}
    \KeywordTok{this}\NormalTok{.}\FunctionTok{distance} \NormalTok{+= }\DecValTok{1}\NormalTok{;}
  \NormalTok{\};}
\NormalTok{\}}
\end{Highlighting}
\end{Shaded}

The above example is silly, but it shows the point that a class can
implement one or more interfaces. Now if the class does not provide
implementations for any of the interfaces, the compiler will complain.
For example, if we leave out the \texttt{distance} instance variable,
the compiler will print out the following error:

\begin{quote}
error TS2420: Class `Car' incorrectly implements interface `ICarProps'.
Property `distance' is missing in type `Car'.
\end{quote}

\subsubsection{Inheritance}\label{inheritance}

In Object-oriented programming, a class can inherit from another class
which helps to define shared attributes and methods among objects.
Although this pattern is very useful, it should be used cautiously as it
can lead to code that is hard to maintain. You can learn more about
classical inheritance and prototypical inheritance by watching Eric
Elliot's \href{https://www.youtube.com/watch?v=lKCCZTUx0sI}{talk} at
O'Reilly's Fluent Conference. The project files for this section are in
\href{https://github.com/st32lth/angular2-intro/tree/master/project-files/typescript/classes/inheritance}{\textbf{\texttt{angular2-intro/project-files/typescript/classes/inheritance}}}.

Let's get started by creating a base class called \texttt{Vehicle}. This
class is going to be the base class for other classes that we create
later.

\begin{Shaded}
\begin{Highlighting}[numbers=left,,]
\CommentTok{// Vehicle.ts}
\NormalTok{export }\KeywordTok{class} \NormalTok{Vehicle \{}
  \FunctionTok{constructor}\NormalTok{( }\KeywordTok{private} \NormalTok{_name: string = 'Vehicle',}
               \KeywordTok{private} \NormalTok{_distance: number = }\DecValTok{0} \NormalTok{) \{ \}}
  \NormalTok{get }\FunctionTok{distance}\NormalTok{(): number \{ }\KeywordTok{return} \KeywordTok{this}\NormalTok{.}\FunctionTok{_distance}\NormalTok{; \}}
  \NormalTok{set }\FunctionTok{distance}\NormalTok{(newDistance: number) \{ }\KeywordTok{this}\NormalTok{.}\FunctionTok{_distance} \NormalTok{= newDistance; \}}
  \NormalTok{get }\FunctionTok{name}\NormalTok{(): string \{ }\KeywordTok{return} \KeywordTok{this}\NormalTok{.}\FunctionTok{_name}\NormalTok{;\}}
  \NormalTok{set }\FunctionTok{name}\NormalTok{(newName: string) \{ }\KeywordTok{this}\NormalTok{.}\FunctionTok{_name} \NormalTok{= newName; \}}
  \FunctionTok{move}\NormalTok{() \{ }\KeywordTok{this}\NormalTok{.}\FunctionTok{distance} \NormalTok{+= }\DecValTok{1} \NormalTok{\}}
  \FunctionTok{toString}\NormalTok{() \{ }\KeywordTok{return} \KeywordTok{this}\NormalTok{.}\FunctionTok{_name}\NormalTok{; \}}
\NormalTok{\}}
\end{Highlighting}
\end{Shaded}

There is nothing special in this class. We are just creating a class
that has two private properties (name, distance) and we are creating the
setters and getters for them. Additionally, we are defining the
\texttt{toString} method that JavaScript internally calls in ``textual
contexts''. The constructor is the most notable of all the other
methods:

\begin{itemize}
\tightlist
\item
  It sets the \texttt{name} property to ``Vehicle'' for all the
  instances
\item
  It also sets the \texttt{distance} property to 0.
\end{itemize}

This means that when a class extends the \texttt{Vehicle} class, it will
have to call the constructor of \texttt{Vehicle} using the
\texttt{super} keyword. Let's do that now by creating two classes called
\texttt{Car} and \texttt{Truck} that inherit from the \texttt{Vehicle}
class:

\textbf{\texttt{cars.ts}}

\begin{Shaded}
\begin{Highlighting}[numbers=left,,]
\KeywordTok{import \{Vehicle\} from './vehicle';}
\NormalTok{export }\KeywordTok{class} \NormalTok{Car }\KeywordTok{extends} \NormalTok{Vehicle \{}
  \FunctionTok{constructor}\NormalTok{(name?: string) \{}
    \KeywordTok{super}\NormalTok{();}
    \KeywordTok{this}\NormalTok{.}\FunctionTok{name} \NormalTok{= name || 'Car';}
  \NormalTok{\}}
\NormalTok{\}}
\NormalTok{export }\KeywordTok{class} \NormalTok{Truck }\KeywordTok{extends} \NormalTok{Vehicle \{}
  \FunctionTok{constructor}\NormalTok{(name?: string) \{}
    \KeywordTok{super}\NormalTok{();}
    \KeywordTok{this}\NormalTok{.}\FunctionTok{name} \NormalTok{= name || 'Truck';}
  \NormalTok{\}}
\NormalTok{\}}
\end{Highlighting}
\end{Shaded}

\begin{itemize}
\tightlist
\item
  The \texttt{Car} class and the \texttt{Truck} class both look almost
  identical. They both inherit from the \texttt{Vehicle} using the
  \texttt{extends} keyword.
\item
  They both call the \texttt{Vehicle}'s constructor in their own
  constructor method before implementing their own:
  \texttt{constructor(name?:\ string)\ \{\ super();\ \}}
\item
  They both take an optional \texttt{name} property to set the name of
  the vehicle. If not name is provided, it will be set to either `Car'
  or `Truck'
\end{itemize}

Now let's create the \texttt{main} file and run the file:

\begin{Shaded}
\begin{Highlighting}[numbers=left,,]
\KeywordTok{import \{Car, Truck\} from './cars';}

\CommentTok{/**}
\CommentTok{ * Creating a new car from `Car`}
\CommentTok{ */}
\DataTypeTok{const} \NormalTok{car = }\KeywordTok{new} \FunctionTok{Car}\NormalTok{();}
\NormalTok{console.}\FunctionTok{log}\NormalTok{(car.}\FunctionTok{name}\NormalTok{);}
\NormalTok{car.}\FunctionTok{distance} \NormalTok{= }\DecValTok{5}\NormalTok{;}
\NormalTok{car.}\FunctionTok{move}\NormalTok{();}
\NormalTok{car.}\FunctionTok{move}\NormalTok{();}
\NormalTok{console.}\FunctionTok{log}\NormalTok{(car.}\FunctionTok{distance}\NormalTok{);}
\CommentTok{/**}
\CommentTok{ * Creating a new Truck.}
\CommentTok{ */}
\DataTypeTok{const} \NormalTok{truck = }\KeywordTok{new} \FunctionTok{Truck}\NormalTok{();}
\NormalTok{console.}\FunctionTok{log}\NormalTok{(truck.}\FunctionTok{name}\NormalTok{);}
\end{Highlighting}
\end{Shaded}

\begin{itemize}
\tightlist
\item
  On line 1 we are importing the \texttt{Car} and the \texttt{Truck}
  class.
\item
  and then we create a \texttt{Car} and \texttt{Truck} instance and log
  their names and distance to the console.
\end{itemize}

Run the build task (command + shift + b) and run the file (F5) and you
should see the output:

\begin{verbatim}
node --debug-brk=7394 --nolazy output/main.js
Debugger listening on port 7394
Car
7
Truck
\end{verbatim}

You can play around with the code above an try passing a string when
instantiating a \texttt{Car} or a \texttt{Truck} to see the name change.

\textbf{TODO}

\begin{itemize}
\tightlist
\item
  \texttt{constructor\ overloading}
\end{itemize}

\subsubsection{Class Decorators}\label{class-decorators}

There are different types of decorators in TypeScript. In this section
we are going to focus on Class Decorators.

\textbf{TODO}

\texttt{add\ content}

\subsection{Modules}\label{modules}

\begin{itemize}
\tightlist
\item
  In TypeScript you can use modules to organize your code, avoid
  polluting the global space, and expose functionalities for others to
  use.
\item
  Multiple modules can be defined in the same file. However, it makes
  more sense to keep on module per file
\item
  If you want, you can split a single module across multiple files
\item
  If you decide to split a module across different files, this is how
  you would do it:

  \begin{itemize}
  \tightlist
  \item
    Create the module file: \texttt{mymodule.ts} and declare your module
    there: \texttt{module\ MyModule\ \{\}}
  \item
    Create another file: \texttt{mymodule.ext1.ts} and on top of the
    file add:
    \texttt{///\ \textless{}reference\ path="mymodule.ts"\ /\textgreater{}}.
    Then in the file, you can use the same name of the module and add
    more stuff to it:
    \texttt{module\ MyModule\ \{\ //\ other\ stuff...\ \}}
  \item
    Then in your main file, you need two things on top of the file:

    \begin{itemize}
    \tightlist
    \item
      \texttt{///\ \textless{}reference\ path="mymodule.ts"\ /\textgreater{}}
    \item
      \texttt{///\ \textless{}reference\ path="mymodule.ext1.ts"\ /\textgreater{}}
    \end{itemize}
  \item
    Then, you can use the name of your module to refer to the symbols
    defined: \texttt{MyModule.something},
    \texttt{MyModule.somethingElse}
  \end{itemize}
\item
  TypeScript has two system: one used internally and the other used
  externally
\item
  External modules are used if your app uses CommonJS or AMD modules.
  Otherwise, you can use TypeScript's internal module system
\item
  Using TypeScript's internal module system, you can:

  \begin{itemize}
  \tightlist
  \item
    use the \texttt{module} keyword to define a module:
    \texttt{module\ MyModule\ \{\ ...\ \}}
  \item
    split modules into different files that contribute to a single
    module
  \item
    use the
    \texttt{///\ \textless{}reference\ path="File.ts"\ /\textgreater{}}
    tag to tell the compiler how files are related to each other when
    modules are split across files
  \end{itemize}
\item
  Using TypeScript's external module system:

  \begin{itemize}
  \tightlist
  \item
    you cannot use the \texttt{module} keyword. The \texttt{module}
    keyword is used only by the internal module system.
  \item
    instead of the \texttt{reference} tag, you can use the
    \texttt{import} keyword to define the relationship between modules
  \item
    you can import symbols using the file name:
    \texttt{import\ mymodule\ =\ require(\textquotesingle{}mymodule\textquotesingle{})}
  \end{itemize}
\end{itemize}

The project files for this chapter are in
\href{https://github.com/st32lth/angular2-intro/tree/master/project-files/typescript/modules}{\textbf{\texttt{angular2-intro/project-files/typescript/modules}}}.

\subsubsection{Simple Module}\label{simple-module}

Let's create a simple module that contains two classes. The first class
is a vehicle class and the second is a car class that inherits from the
vehicle class. Then we are going to expose the car class to the outside
world and import it from another file. The project files for this
section are in
\href{https://github.com/st32lth/angular2-intro/tree/master/project-files/typescript/modules/basic-module}{\textbf{\texttt{angular2-intro/project-files/typescript/modules/basic-module}}}.

First, create the \texttt{main.ts} file and copy paste the following:

\textbf{\texttt{main.ts}}

\begin{Shaded}
\begin{Highlighting}[numbers=left,,]
\NormalTok{module MyModule \{}
  \KeywordTok{class} \NormalTok{Vehicle \{}
    \FunctionTok{constructor} \NormalTok{(}\KeywordTok{public} \NormalTok{name: string = 'Vehicle', }\KeywordTok{private} \NormalTok{_distance: number = }\DecValTok{0}\NormalTok{) \{\}}
    \NormalTok{get }\FunctionTok{distance}\NormalTok{():number \{ }\KeywordTok{return} \KeywordTok{this}\NormalTok{.}\FunctionTok{_distance}\NormalTok{; \}}
    \NormalTok{set }\FunctionTok{distance}\NormalTok{(newDistance: number) \{ }\KeywordTok{this}\NormalTok{.}\FunctionTok{_distance} \NormalTok{= newDistance; \}}
    \FunctionTok{move}\NormalTok{() \{ }\KeywordTok{this}\NormalTok{.}\FunctionTok{distance} \NormalTok{+= }\DecValTok{1} \NormalTok{\}}
  \NormalTok{\}}
\NormalTok{\}}
\end{Highlighting}
\end{Shaded}

\begin{itemize}
\tightlist
\item
  On line 1 we are defining the module called \texttt{MyModule}.
\item
  Inside this module we have defined a class called \texttt{Vehicle}
  that has a distance property and a setter and getter.
\end{itemize}

Now we want to create a class and export it so that it can be imported
by others:

\textbf{\texttt{main.ts}}

\begin{Shaded}
\begin{Highlighting}[numbers=left,,]
\NormalTok{module MyModule \{}
  \KeywordTok{class} \NormalTok{Vehicle \{}
    \FunctionTok{constructor} \NormalTok{(}\KeywordTok{public} \NormalTok{name: string = 'Vehicle', }\KeywordTok{private} \NormalTok{_distance: number = }\DecValTok{0}\NormalTok{) \{\}}
    \NormalTok{get }\FunctionTok{distance}\NormalTok{():number \{ }\KeywordTok{return} \KeywordTok{this}\NormalTok{.}\FunctionTok{_distance}\NormalTok{; \}}
    \NormalTok{set }\FunctionTok{distance}\NormalTok{(newDistance: number) \{ }\KeywordTok{this}\NormalTok{.}\FunctionTok{_distance} \NormalTok{= newDistance; \}}
    \FunctionTok{move}\NormalTok{() \{ }\KeywordTok{this}\NormalTok{.}\FunctionTok{distance} \NormalTok{+= }\DecValTok{1} \NormalTok{\}}
  \NormalTok{\}}
  \CommentTok{// -> adding the car class}
  \NormalTok{export }\KeywordTok{class} \NormalTok{Car }\KeywordTok{extends} \NormalTok{Vehicle \{}
    \FunctionTok{constructor} \NormalTok{(}\KeywordTok{public} \NormalTok{name: string = 'Car') \{}
      \KeywordTok{super}\NormalTok{();}
    \NormalTok{\}}
  \NormalTok{\}}
\NormalTok{\}}
\end{Highlighting}
\end{Shaded}

\begin{itemize}
\tightlist
\item
  On line 9 we are using the \texttt{export} keyword to indicate that
  the \texttt{Car} class is exposed and can be used by others.
\end{itemize}

Now, let's create a car using the \texttt{Car} class defined in the
\texttt{MyModule} module:

\begin{Shaded}
\begin{Highlighting}[numbers=left,,]
\DataTypeTok{const} \NormalTok{mycar = }\KeywordTok{new} \NormalTok{MyModule.}\FunctionTok{Car}\NormalTok{('My Car');}
\NormalTok{console.}\FunctionTok{log}\NormalTok{(mycar.}\FunctionTok{name}\NormalTok{);}
\end{Highlighting}
\end{Shaded}

Note that we accessed the \texttt{Car} class using the \texttt{MyModule}
symbol: \texttt{MyModule.Car}. Now we can split up the module into its
own file and import it into the main file. Let's create a file called
\texttt{MyModule.ts} and move the module definition to that file. Now in
our main file we are just going to import the module and use the car
class from it.

\textbf{\texttt{main.ts}}

\begin{Shaded}
\begin{Highlighting}[numbers=left,,]
\CommentTok{/// <reference path="MyModule.ts" />}
\DataTypeTok{const} \NormalTok{mycar = }\KeywordTok{new} \NormalTok{MyModule.}\FunctionTok{Car}\NormalTok{('My Car');}
\NormalTok{console.}\FunctionTok{log}\NormalTok{(mycar.}\FunctionTok{name}\NormalTok{);}
\end{Highlighting}
\end{Shaded}

Note that we can create an alias to the \texttt{MyModule} using
\texttt{import\ AliasName\ =\ MyModule}. Now you can reference the
module name with \texttt{AliasName}:

\begin{Shaded}
\begin{Highlighting}[numbers=left,,]
\CommentTok{/// <reference path="MyModule.ts" />}
\KeywordTok{import AliasName = MyModule;}
\DataTypeTok{const} \NormalTok{mycar = }\KeywordTok{new} \NormalTok{AliasName.}\FunctionTok{Car}\NormalTok{('My Car');}
\NormalTok{console.}\FunctionTok{log}\NormalTok{(mycar.}\FunctionTok{name}\NormalTok{);}
\end{Highlighting}
\end{Shaded}

Now if we run this in debug mode, the compiler will complain that it
can't find the \texttt{MyModule} reference. Because of that we need to
make some changes to our config files. First, we are going to add the
\texttt{out} property in the \texttt{tsconfig.json} file. This will tell
the compiler to compile all the files into a single file:

\begin{verbatim}
"out": "output/run.js",
\end{verbatim}

So our \texttt{tsconfig.json} file will look like this:

\begin{verbatim}
{
  "compilerOptions": {
    "experimentalDecorators": true,
    "emitDecoratorMetadata": true,
    "module": "commonjs",
    "target": "es5",
    "sourceMap": true,
    "outDir": "output",
    "out": "output/run.js",
    "watch": true
  }
}
\end{verbatim}

Now if you run the build, you should see that all the project has been
compiled into \texttt{output/run.js}. In addition to the
\texttt{tsconfig.json} file, we are going to update the
\texttt{launch.json} file and add a new configuration field:

\begin{verbatim}
{
  "name": "TS All Debugger",
  "type": "node",
  "program": "output/run.js",
  "stopOnEntry": false,
  "sourceMaps": true
}
\end{verbatim}

Now we should be able to use the debugger and put breakpoints in our
TypeScript files. Select \texttt{TS\ All\ Debugger} from the debugger
dropdown and run the debugger and it should stop if you put a breakpoint
in any of your TypeScript files.

\textbf{NOTE} Using the configuration files above we can compile all the
TypeScript files into a single JavaScript file. But sometimes that is
not what you want. Be aware that using the above configuration you will
not get an output for each TypeScript file.

\subsubsection{Splitting Internal
Modules}\label{splitting-internal-modules}

Internal modules in TypeScript are open ended. This means that you can
define a module with the same name in different files and keep adding to
it. This is also known as merging. In this section we are going to
demonstrate merging multiple files that contribute to a single module
called \texttt{Merged}. The project files for this section are in
\href{https://github.com/st32lth/angular2-intro/tree/master/project-files/typescript/modules/merged-module}{\textbf{\texttt{angular2-intro/project-files/typescript/modules/merged-module}}}.

First, we are going to make two files: \texttt{A.ts} and \texttt{B.ts}.
In each file we are going to define the \texttt{Merged} module:

\begin{Shaded}
\begin{Highlighting}[numbers=left,,]
\CommentTok{// A.ts}
\NormalTok{module Merged \{}
  \DataTypeTok{const} \NormalTok{name = 'File A'; }\CommentTok{// not exported}
  \NormalTok{export }\KeywordTok{class} \NormalTok{Door \{}
    \FunctionTok{constructor} \NormalTok{(}\KeywordTok{private} \NormalTok{_color = 'white') \{\}}
    \NormalTok{get }\FunctionTok{color}\NormalTok{() \{ }\KeywordTok{return} \KeywordTok{this}\NormalTok{.}\FunctionTok{_color}\NormalTok{; \}}
    \NormalTok{set }\FunctionTok{color}\NormalTok{(newColor) \{ }\KeywordTok{this}\NormalTok{.}\FunctionTok{_color} \NormalTok{= newColor; \}}
  \NormalTok{\}}
\NormalTok{\}}
\end{Highlighting}
\end{Shaded}

and then the \texttt{B.ts} file:

\begin{Shaded}
\begin{Highlighting}[numbers=left,,]
\CommentTok{// B.ts}
\NormalTok{module Merged \{}
  \DataTypeTok{const} \NormalTok{name = 'File B'; }\CommentTok{// not exported}
  \NormalTok{export }\KeywordTok{class} \NormalTok{Car \{}
    \FunctionTok{constructor}\NormalTok{(}\KeywordTok{public} \NormalTok{distance = }\DecValTok{0}\NormalTok{) \{\}}
    \FunctionTok{move} \NormalTok{() \{}\KeywordTok{this}\NormalTok{.}\FunctionTok{distance} \NormalTok{+= }\DecValTok{1}\NormalTok{;\}}
  \NormalTok{\}}
\NormalTok{\}}
\end{Highlighting}
\end{Shaded}

We just created two files called \texttt{A.ts} and \texttt{B.ts} and
each file we defined the \texttt{Merged} module and added a class to
each and exported it. Now we are going to make the \texttt{main.ts} file
and reference these two files:

\begin{Shaded}
\begin{Highlighting}[numbers=left,,]
\CommentTok{// main.ts}
\CommentTok{/// <reference path="./A.ts" />}
\CommentTok{/// <reference path="./B.ts" />}
\end{Highlighting}
\end{Shaded}

And now we can use the classes defined in the \texttt{Merged} module,
that is the \texttt{Car} and the \texttt{Door} class:

\begin{Shaded}
\begin{Highlighting}[numbers=left,,]
\CommentTok{/// <reference path="./A.ts" />}
\CommentTok{/// <reference path="./B.ts" />}
\DataTypeTok{const} \NormalTok{car: Merged.}\FunctionTok{Car} \NormalTok{= }\KeywordTok{new} \NormalTok{Merged.}\FunctionTok{Car}\NormalTok{();}
\DataTypeTok{const} \NormalTok{door: Merged.}\FunctionTok{Door} \NormalTok{= }\KeywordTok{new} \NormalTok{Merged.}\FunctionTok{Door}\NormalTok{();}
\NormalTok{door.}\FunctionTok{color} \NormalTok{= 'blue';}
\NormalTok{car.}\FunctionTok{move}\NormalTok{();}
\NormalTok{car.}\FunctionTok{move}\NormalTok{();}
\NormalTok{console.}\FunctionTok{log}\NormalTok{(car.}\FunctionTok{distance}\NormalTok{);}
\NormalTok{console.}\FunctionTok{log}\NormalTok{(door.}\FunctionTok{color}\NormalTok{);}
\end{Highlighting}
\end{Shaded}

if you run the build task (command + shift + b) and hit F5 you should
see the following output:

\begin{verbatim}
node --debug-brk=19237 --nolazy output/run.js
Debugger listening on port 19237
2
blue
\end{verbatim}

\subsubsection{External Modules}\label{external-modules}

In addition to TypeScript's internal module system, you can use external
modules as well. In this section we are going to demonstrate how you can
use external modules in TypeScript. The project files for this section
are in
\href{https://github.com/st32lth/angular2-intro/tree/master/project-files/typescript/modules/external-module}{\textbf{\texttt{angular2-intro/project-files/typescript/modules/external-module}}}.

Let's say I have a JavaScript Node module defined in CommonJS format in
a file called \texttt{common.js}:

\begin{Shaded}
\begin{Highlighting}[numbers=left,,]
\CommentTok{// common.js}
\VariableTok{module}\NormalTok{.}\AttributeTok{exports} \OperatorTok{=} \KeywordTok{function} \NormalTok{() }\OperatorTok{\{}
  \KeywordTok{this}\NormalTok{.}\AttributeTok{name} \OperatorTok{=} \StringTok{'CommonJS Module'}\OperatorTok{;}
\OperatorTok{\};}
\end{Highlighting}
\end{Shaded}

In order to import this we need to do two things: first, we need to
install Node's Type Definitions. Then we need to require the module. To
install Node's Type Definitions run the following the terminal in the
root of your project:

\begin{verbatim}
tsd install node --save
\end{verbatim}

Now you should see a folder called \texttt{typings} containing the type
definitions. Now that we have Node's type definitions, let's add a
reference to it on top of \texttt{main.ts}:

\begin{Shaded}
\begin{Highlighting}[numbers=left,,]
\CommentTok{// main.ts}
\CommentTok{/// <reference path="./typings/node/node.d.ts" />}
\end{Highlighting}
\end{Shaded}

and then we are going to require the module and log it to the console:

\begin{Shaded}
\begin{Highlighting}[numbers=left,,]
\CommentTok{// main.ts}
\CommentTok{/// <reference path="./typings/node/node.d.ts" />}
\DataTypeTok{const} \NormalTok{common = }\FunctionTok{require}\NormalTok{('./common');}
\NormalTok{console.}\FunctionTok{log}\NormalTok{(}\FunctionTok{common}\NormalTok{()); }\CommentTok{// --> CommonJS Module}
\end{Highlighting}
\end{Shaded}

After running the build task ( command + shift + b ), and running the
file (F+5) you should see the following output:

\begin{verbatim}
node --debug-brk=32221 --nolazy run.js 
Debugger listening on port 32221
CommonJS Modules
\end{verbatim}

\textbf{Note} the configuration files that we are using:

\textbf{\texttt{tsconfig.json}}

\begin{verbatim}
{
  "compilerOptions": {
    "experimentalDecorators": true,
    "emitDecoratorMetadata": true,
    "module": "commonjs",
    "target": "es5",
    "sourceMap": true,
    "outDir": "output",
    "out": "run.js",
    "watch": true
  }
}
\end{verbatim}

\textbf{\texttt{launch.json}}

\begin{verbatim}
{
  "version": "0.1.0",
  "configurations": [
    {
      "name": "TS All Debugger",
      "type": "node",
      "program": "./run.js",
      "stopOnEntry": false,
      "sourceMaps": true
    }
  ]
}
\end{verbatim}

\subsection{Decorators}\label{decorators}

\begin{itemize}
\tightlist
\item
  Decorators can be used to add additional properties and methods to
  existing objects.
\item
  Decorators are a declarative way to add metadata to code.
\item
  There are four decorators: ClassDecorator, PropertyDecorator,
  MethodDecorator, ParameterDecorator
\item
  TypeScript supports decorators and does not know about Angular's
  specific annotations.
\item
  Angular provides annotations that are made with decorators behind the
  scenes
\end{itemize}

\subsubsection{Method Decorators}\label{method-decorators}

Goals: - make a method decorator called \texttt{log}. - Decorate
\texttt{someMethod} in a class using \texttt{@log}

\begin{Shaded}
\begin{Highlighting}[numbers=left,,]
\KeywordTok{class} \NormalTok{SomeClass \{}
  \FunctionTok{@log}
  \FunctionTok{someMethod}\NormalTok{(n: number) \{}
    \KeywordTok{return} \NormalTok{n * }\DecValTok{2}\NormalTok{;}
  \NormalTok{\}}
\NormalTok{\}}
\end{Highlighting}
\end{Shaded}

In the usage, \texttt{someMethod} has been decorated with \texttt{log}
using the \texttt{@} symbol. \texttt{@log} is decorating
\texttt{someMethod} because it is placed right before the method.

\begin{itemize}
\tightlist
\item
  Decorator Implementation:
\end{itemize}

\begin{Shaded}
\begin{Highlighting}[numbers=left,,]
\NormalTok{function }\FunctionTok{log}\NormalTok{(target: Function, key: string, value: any) \{}
  \KeywordTok{return} \NormalTok{\{}
    \NormalTok{value: }\FunctionTok{function} \NormalTok{(...}\FunctionTok{args}\NormalTok{: any[]) \{}
      \NormalTok{var a = args.}\FunctionTok{map}\NormalTok{(a => JSON.}\FunctionTok{stringify}\NormalTok{(a)).}\FunctionTok{join}\NormalTok{();}
      \NormalTok{var result = value.}\FunctionTok{value}\NormalTok{.}\FunctionTok{apply}\NormalTok{(}\KeywordTok{this}\NormalTok{, args);}
      \NormalTok{var r = JSON.}\FunctionTok{stringify}\NormalTok{(result);}
      \NormalTok{console.}\FunctionTok{log}\NormalTok{(`Call: $\{key\}($\{a\}) => $\{r\}`);}
      \KeywordTok{return} \NormalTok{result;}
    \NormalTok{\}}
  \NormalTok{\};}
\NormalTok{\}}
\end{Highlighting}
\end{Shaded}

A method decorators takes a 3 arguments:

\begin{itemize}
\tightlist
\item
  \texttt{target}: the method being decorated.
\item
  \texttt{key}: the name of the method being decorated.
\item
  \texttt{value}: a property descriptor of the given property if it
  exists on the object, undefined otherwise. The property descriptor is
  obtained by invoking the \texttt{Object.getOwnPropertyDescriptor}
  function.
\end{itemize}

\textbf{TODO}

\begin{itemize}
\tightlist
\item
  Add decorator content for each type.
\end{itemize}

\section{Angular Basics}\label{angular-basics}

This chapter will walk you through the basics of Angular2. We will start
by looking at the basics of components, and then we move onto pipes,
events and other concepts. By the end of the chapter you should have a
basic understanding of the new concepts in Angular2.

The goal of this chapter is to get your feet wet without scaring you
with a lot of details. Don't worry, there will be a lot coming in the
later chapters.

\hypertarget{using-angular-project-files}{\subsection{Using Angular
Project Files}\label{using-angular-project-files}}

In order to run the project files, you need to do two things:

\begin{itemize}
\item
  First, install the server dependencies and run the server in the root
  of the code repo:

\begin{verbatim}
npm i && npm start
\end{verbatim}

  After the dependencies are installed, it will open up the browser at
  port 8080.
\item
  The next step is to install the dependencies for angular examples. Go
  to \texttt{project-files/angular-examples} and install the
  dependencies:

\begin{verbatim}
cd project-files/angular-examples && npm i
\end{verbatim}
\end{itemize}

After following the steps above, you should be able to see the examples
in the browser. For example, if you want to see the
\texttt{basic-component} example, you can go to the following url:

\begin{verbatim}
http://localhost:8080/project-files/angular-examples/basic-component/index.html
\end{verbatim}

\subsection{Components}\label{components}

Components are at the heart of Angular. The main idea is that you break
down your application into different cohesive components and let the
components handle the rest. Every component has a controller defined by
a class and a template defined by html. In addition, a component's job
is to enable the user experience and delegate everything non-trivial to
services.

In this section we are going to write a simple \texttt{HelloAngular}
component, compile it and run it in the browser. In addition, we will
configure VSCode to build the TypeScript files as we go.

Note that there is a lot to talk about components. We are going dive
into components a lot more in later chapters, but for now let's just
keep things simple.

\subsubsection{Project Files}\label{project-files}

The project files for this chapter are in
\textbf{\href{https://github.com/st32lth/angular2-intro/tree/master/project-files/angular-examples/basic-component}{\texttt{angular2-intro/project-files/angular-examples/basic-component}}}
You can either follow along or just look at the final result

In order to run the project files, please refer to the
\protect\hyperlink{using-angular-project-files}{Using Angular Project
Files} section.

\subsubsection{Getting Started}\label{getting-started}

Make a folder on your desktop called \texttt{hello-angular} and navigate
to it:

\begin{verbatim}
mkdir ~/Desktop/hello-angular && cd $_
\end{verbatim}

Start npm in this folder with \texttt{npm\ init} and accept all the
defaults.

After that, add the \texttt{dependencies} and \texttt{devDependencies}
field to your \texttt{package.json} file:

\begin{Shaded}
\begin{Highlighting}[numbers=left,,]
\ErrorTok{"dependencies":} \FunctionTok{\{}
  \DataTypeTok{"angular2"}\FunctionTok{:} \StringTok{"^2.0.0-beta.1"}\FunctionTok{,}
  \DataTypeTok{"es6-promise"}\FunctionTok{:} \StringTok{"^3.0.2"}\FunctionTok{,}
  \DataTypeTok{"es6-shim"}\FunctionTok{:} \StringTok{"^0.33.3"}\FunctionTok{,}
  \DataTypeTok{"reflect-metadata"}\FunctionTok{:} \StringTok{"0.1.2"}\FunctionTok{,}
  \DataTypeTok{"rxjs"}\FunctionTok{:} \StringTok{"5.0.0-beta.0"}\FunctionTok{,}
  \DataTypeTok{"zone.js"}\FunctionTok{:} \StringTok{"0.5.10"}
\FunctionTok{\}}\ErrorTok{,}
\ErrorTok{"devDependencies":} \FunctionTok{\{}
  \DataTypeTok{"systemjs"}\FunctionTok{:} \StringTok{"^0.19.16"}
\FunctionTok{\}}
\end{Highlighting}
\end{Shaded}

your \texttt{package.json} file should look something like the
follwoing:

\begin{Shaded}
\begin{Highlighting}[numbers=left,,]
\FunctionTok{\{}
  \DataTypeTok{"name"}\FunctionTok{:} \StringTok{"hello-angular"}\FunctionTok{,}
  \DataTypeTok{"version"}\FunctionTok{:} \StringTok{"1.0.0"}\FunctionTok{,}
  \DataTypeTok{"description"}\FunctionTok{:} \StringTok{""}\FunctionTok{,}
  \DataTypeTok{"main"}\FunctionTok{:} \StringTok{"index.js"}\FunctionTok{,}
  \DataTypeTok{"scripts"}\FunctionTok{:} \FunctionTok{\{}
    \DataTypeTok{"test"}\FunctionTok{:} \StringTok{"echo }\CharTok{\textbackslash{}"}\StringTok{Error: no test specified}\CharTok{\textbackslash{}"}\StringTok{ && exit 1"}
  \FunctionTok{\},}
  \DataTypeTok{"author"}\FunctionTok{:} \StringTok{"Stealth <st32lth@gmail.com> (http://github.com/st32lth)"}\FunctionTok{,}
  \DataTypeTok{"license"}\FunctionTok{:} \StringTok{"ISC"}\FunctionTok{,}
  \DataTypeTok{"dependencies"}\FunctionTok{:} \FunctionTok{\{}
    \DataTypeTok{"angular2"}\FunctionTok{:} \StringTok{"^2.0.0-beta.1"}\FunctionTok{,}
    \DataTypeTok{"es6-promise"}\FunctionTok{:} \StringTok{"^3.0.2"}\FunctionTok{,}
    \DataTypeTok{"es6-shim"}\FunctionTok{:} \StringTok{"^0.33.3"}\FunctionTok{,}
    \DataTypeTok{"reflect-metadata"}\FunctionTok{:} \StringTok{"0.1.2"}\FunctionTok{,}
    \DataTypeTok{"rxjs"}\FunctionTok{:} \StringTok{"5.0.0-beta.0"}\FunctionTok{,}
    \DataTypeTok{"zone.js"}\FunctionTok{:} \StringTok{"0.5.10"}
  \FunctionTok{\},}
  \DataTypeTok{"devDependencies"}\FunctionTok{:} \FunctionTok{\{}
    \DataTypeTok{"systemjs"}\FunctionTok{:} \StringTok{"^0.19.16"}
  \FunctionTok{\}}
\FunctionTok{\}}
\end{Highlighting}
\end{Shaded}

Then run \texttt{npm\ i} to install the dependencies.

After all the dependencies are installed, start VSCode in this folder
with \texttt{code\ .}

Then create a \texttt{index.html} file in the root of the project and
put in the following:

\textbf{\texttt{index.html}}

\begin{Shaded}
\begin{Highlighting}[numbers=left,,]
\KeywordTok{<html>}
\KeywordTok{<head>}
  \KeywordTok{<title>}\NormalTok{Hello Angular}\KeywordTok{</title>}

  \KeywordTok{<script}\OtherTok{ src=}\StringTok{"/node_modules/angular2/bundles/angular2-polyfills.js"}\KeywordTok{></script>}
  \KeywordTok{<script}\OtherTok{ src=}\StringTok{"/node_modules/systemjs/dist/system.src.js"}\KeywordTok{></script>}
  \KeywordTok{<script}\OtherTok{ src=}\StringTok{"/node_modules/rxjs/bundles/Rx.js"}\KeywordTok{></script>}
  \KeywordTok{<script}\OtherTok{ src=}\StringTok{"/node_modules/angular2/bundles/angular2.dev.js"}\KeywordTok{></script>}

  \CommentTok{<!-- add systemjs settings later -->}

\KeywordTok{</head>}

\KeywordTok{<body>}
  \CommentTok{<!-- add app stuff later -->}
\KeywordTok{</body>}

\KeywordTok{</html>}
\end{Highlighting}
\end{Shaded}

This loads all the necessary scripts that we need to run Angular in the
browser.

\textbf{Note}

If you need to support older browsers, you need to include the
\texttt{es6-shims} before everything else:

\begin{Shaded}
\begin{Highlighting}[numbers=left,,]
\KeywordTok{<script}\OtherTok{ src=}\StringTok{"/node_modules/es6-shim/es6-shim.js"}\KeywordTok{></script>}
\end{Highlighting}
\end{Shaded}

\subsubsection{Making the Component}\label{making-the-component}

Let's start by making the \texttt{main.ts} file in the root of the
project. In this file we are going to define the main component called
\texttt{HelloAngular} and then bootstrap the app with it:

\textbf{\texttt{main.ts}}

\begin{Shaded}
\begin{Highlighting}[numbers=left,,]
\KeywordTok{import \{Component, OnInit\} from 'angular2/core';}
\KeywordTok{import \{bootstrap\} from 'angular2/platform/browser';}

\FunctionTok{@Component}\NormalTok{(\{}
  \NormalTok{selector: 'app',}
  \NormalTok{styles: [`h1 \{ line-height: 100vh; text-align: center \}`],}
  \NormalTok{template: `<h1>\{\{ name \}\}</h1>`}
\NormalTok{\})}
\KeywordTok{class} \NormalTok{HelloAngular }\KeywordTok{implements} \NormalTok{OnInit \{}
  \NormalTok{name: string;}
  \FunctionTok{constructor}\NormalTok{() \{ }\KeywordTok{this}\NormalTok{.}\FunctionTok{name} \NormalTok{= 'Hello Angular'; \}}
  \FunctionTok{ngOnInit}\NormalTok{() \{ console.}\FunctionTok{log}\NormalTok{('component linked'); \}}
\NormalTok{\}}

\FunctionTok{bootstrap}\NormalTok{(HelloAngular, []);}
\end{Highlighting}
\end{Shaded}

\begin{itemize}
\tightlist
\item
  On line 1 we are importing the \texttt{component} meta data
  (annotation) and the \texttt{onInit} interface.
\item
  On line 2 we are loading the \texttt{bootstrap} method that bootstraps
  the app given a component.
\item
  On line 4, we are defining a component using the \texttt{component}
  decorator. The \texttt{@component} is technically a class decorator
  because it precedes the \texttt{HelloAngular} class definition.
\item
  On line 5, we are telling angular to look out for the \texttt{app}
  tag. So when Angular looks at the html and comes across the
  \texttt{\textless{}app\textgreater{}\textless{}/app\textgreater{}}
  tag, it is going to load the template (on line 6) and instantiates the
  class for it (defined on line 9).
\item
  On line 9, we are defining a class called \texttt{HelloAngular} that
  defines the logic of the component. And for fun, we are implementing
  the \texttt{OnInit} interface to log something to the console when the
  component is ready with its data. We will learn more about the
  lifeCycle hooks later.
\item
  Last but not least, we call the \texttt{bootstrap} method with the
  \texttt{HelloAngular} class as the first argument to bootstrap the app
  with the \texttt{HelloAngular} component.
\end{itemize}

\subsubsection{Compiling the Component}\label{compiling-the-component}

Now we need to compile the file to JavaScript. We can do it from the
terminal, but let's stick to VSCode. In order to that, we need to make
two config files:

\begin{enumerate}
\def\labelenumi{\arabic{enumi}.}
\item
  First is the standard
  \href{http://json.schemastore.org/tsconfig}{\texttt{tsconfig.json}}
  file
\item
  And the \texttt{tasks.json} file for VSCode to do the compiling
\end{enumerate}

Create the \texttt{tsconfig.json} file in the root of the project and
put in the following:

\textbf{\texttt{tsconfig.json}}

\begin{Shaded}
\begin{Highlighting}[numbers=left,,]
\FunctionTok{\{}
  \DataTypeTok{"compilerOptions"}\FunctionTok{:} \FunctionTok{\{}
    \DataTypeTok{"target"}\FunctionTok{:} \StringTok{"es5"}\FunctionTok{,}
    \DataTypeTok{"module"}\FunctionTok{:} \StringTok{"system"}\FunctionTok{,}
    \DataTypeTok{"moduleResolution"}\FunctionTok{:} \StringTok{"node"}\FunctionTok{,}
    \DataTypeTok{"sourceMap"}\FunctionTok{:} \KeywordTok{true}\FunctionTok{,}
    \DataTypeTok{"emitDecoratorMetadata"}\FunctionTok{:} \KeywordTok{true}\FunctionTok{,}
    \DataTypeTok{"experimentalDecorators"}\FunctionTok{:} \KeywordTok{true}\FunctionTok{,}
    \DataTypeTok{"removeComments"}\FunctionTok{:} \KeywordTok{false}\FunctionTok{,}
    \DataTypeTok{"noImplicitAny"}\FunctionTok{:} \KeywordTok{false}\FunctionTok{,}
    \DataTypeTok{"outDir"}\FunctionTok{:} \StringTok{"output"}\FunctionTok{,}
    \DataTypeTok{"watch"}\FunctionTok{:} \KeywordTok{true}
  \FunctionTok{\},}
  \DataTypeTok{"exclude"}\FunctionTok{:} \OtherTok{[}
    \StringTok{"node_modules"}
  \OtherTok{]}
\FunctionTok{\}}
\end{Highlighting}
\end{Shaded}

Then create the \texttt{tasks.json} in the \texttt{.vscode} folder in
the root of the project and put in the following:

\textbf{\texttt{.vscode/tasks.json}}

\begin{Shaded}
\begin{Highlighting}[numbers=left,,]
\FunctionTok{\{}
  \DataTypeTok{"version"}\FunctionTok{:} \StringTok{"0.1.0"}\FunctionTok{,}
  \DataTypeTok{"command"}\FunctionTok{:} \StringTok{"tsc"}\FunctionTok{,}
  \DataTypeTok{"showOutput"}\FunctionTok{:} \StringTok{"silent"}\FunctionTok{,}
  \DataTypeTok{"isShellCommand"}\FunctionTok{:} \KeywordTok{true}\FunctionTok{,}
  \DataTypeTok{"problemMatcher"}\FunctionTok{:} \StringTok{"$tsc"}
\FunctionTok{\}}
\end{Highlighting}
\end{Shaded}

\begin{itemize}
\item
  Now we can build the TypeScript files as we work. We just need to
  start the build task with \texttt{command\ +\ shift\ +\ b} or using
  the prompt. If you want to use the prompt do the following:

  \begin{itemize}
  \item
    Use \texttt{command\ +\ shift\ +\ p} to open the prompt
  \item
    Then, type \texttt{\textgreater{}\ run\ build\ task} and hit enter
    to start the build task.
  \end{itemize}
\item
  After you run the build task, you should see an \texttt{output} file
  generated with \texttt{main.js} and the source maps in it.
\item
  The task is watching the files and compiling as you go. To stop the
  task, open the prompt and type:

\begin{verbatim}
> terminate running task
\end{verbatim}
\end{itemize}

\subsubsection{Loading the Component}\label{loading-the-component}

After compiling the component, we need to load it to the
\texttt{index.html} file with \texttt{Systemjs}. Open the
\texttt{index.html} file and replace
\texttt{\textless{}!-\/-\ add\ systemjs\ settings\ later\ -\/-\textgreater{}}
with the following:

\begin{Shaded}
\begin{Highlighting}[numbers=left,,]
\KeywordTok{<script>}
  \VariableTok{System}\NormalTok{.}\AttributeTok{config}\NormalTok{(}\OperatorTok{\{}
    \DataTypeTok{packages}\OperatorTok{:} \OperatorTok{\{}
      \DataTypeTok{output}\OperatorTok{:} \OperatorTok{\{}
        \DataTypeTok{format}\OperatorTok{:} \StringTok{'register'}\OperatorTok{,}
        \DataTypeTok{defaultExtension}\OperatorTok{:} \StringTok{'js'}
      \OperatorTok{\}}
    \OperatorTok{\}}
  \OperatorTok{\}}\NormalTok{)}\OperatorTok{;}
  \VariableTok{System}\NormalTok{.}\AttributeTok{import}\NormalTok{(}\StringTok{'output/main'}\NormalTok{)}
  \NormalTok{.}\AttributeTok{then}\NormalTok{(}\KeywordTok{null}\OperatorTok{,} \VariableTok{console}\NormalTok{.}\VariableTok{error}\NormalTok{.}\AttributeTok{bind}\NormalTok{(console))}\OperatorTok{;}
\OperatorTok{<}\SpecialStringTok{/script>}
\end{Highlighting}
\end{Shaded}

Now we can use our component in the body of the html:

\begin{Shaded}
\begin{Highlighting}[numbers=left,,]
\KeywordTok{<body>}
  \KeywordTok{<app>}\NormalTok{Loading ...}\KeywordTok{</app>}
\KeywordTok{</body>}
\end{Highlighting}
\end{Shaded}

It is finally time to serve the app. You can serve the app in the
current directory using the \texttt{live-server}:

\begin{verbatim}
live-server .
\end{verbatim}

If everything is wired up correctly, you should be able to see the
following:

\begin{figure}[htbp]
\centering
\includegraphics{images/hello-angular.png}
\caption{Running a basic component in the browser}
\end{figure}

\subsubsection{Debugging the component}\label{debugging-the-component}

You can connect chrome's debugger to VSCode using the chrome debugger
extension for Visual Studio Code. See the
\protect\hyperlink{debugging-app-from-vscode}{Debuggin App from VSCode}
section in case you missed to install it. But, assuming that you have
the extension installed, you can debug your app from VSCode. In order to
do that, we need to create a \texttt{launch.json} file in the
\texttt{.vscode} folder:

\begin{verbatim}
touch .vscode/launch.json
\end{verbatim}

After you created the file, put in the following configuration in the
file:

\begin{Shaded}
\begin{Highlighting}[numbers=left,,]
\FunctionTok{\{}
  \DataTypeTok{"version"}\FunctionTok{:} \StringTok{"0.1.0"}\FunctionTok{,}
  \DataTypeTok{"configurations"}\FunctionTok{:} \OtherTok{[}
    \FunctionTok{\{}
      \DataTypeTok{"name"}\FunctionTok{:} \StringTok{"Launch Chrome Debugger"}\FunctionTok{,}
      \DataTypeTok{"type"}\FunctionTok{:} \StringTok{"chrome"}\FunctionTok{,}
      \DataTypeTok{"request"}\FunctionTok{:} \StringTok{"launch"}\FunctionTok{,}
      \DataTypeTok{"url"}\FunctionTok{:} \StringTok{"http://127.0.0.1:8080/"}\FunctionTok{,}
      \DataTypeTok{"sourceMaps"}\FunctionTok{:} \KeywordTok{true}\FunctionTok{,}
      \DataTypeTok{"webRoot"}\FunctionTok{:} \StringTok{"."}\FunctionTok{,}
      \DataTypeTok{"runtimeExecutable"}\FunctionTok{:} \StringTok{"/Applications/Google Chrome.app/Contents/MacOS/Google Chrome"}\FunctionTok{,}
      \DataTypeTok{"runtimeArgs"}\FunctionTok{:} \OtherTok{[}
        \StringTok{"--remote-debugging-port=9222"}\OtherTok{,}
        \StringTok{"--incognito"}
      \OtherTok{]}
    \FunctionTok{\}}
  \OtherTok{]}
\FunctionTok{\}}
\end{Highlighting}
\end{Shaded}

Before running the debugger:

\begin{itemize}
\tightlist
\item
  Make sure that all instances of chrome are closed. It makes it easier
  to run the debugger from VSCode itself.
\item
  Make sure that the \texttt{runtimeExecutable} path is valid. This
  value would be different depending on your OS.
\item
  Make sure that the \texttt{url} value is valid as well. The
  \texttt{url} value has to match the path that you see when you run a
  server serving the files.
\item
  Set a breakpoint on a line in \texttt{main.ts} file and then run the
  debugger under the debugger tab.
\end{itemize}

In order to run the debugger, select \texttt{Launch\ Chrome\ Debugger}
in the dropdown under the debugger tab and either click on the play icon
or hit F5 on the keyboard. After that, an instance of Chrome should be
opened in incognito mode. In order to trigger the debugger just refresh
the page and you should be able to see the debugger pausing in VSCode.
If everything is set up correctly you should be able to see something
like the following screenshot:

\begin{figure}[htbp]
\centering
\includegraphics{images/run-debugger.png}
\caption{Debugging the component in VSCode}
\end{figure}

\subsection{Metadata Classes}\label{metadata-classes}

\begin{itemize}
\tightlist
\item
  Angular uses Metadata to decorate classes, methods and properties.
\item
  The most notable Metadata is the \texttt{@component} Metadata.
\item
  Metadta classes are very convenient and they make it easy to work with
  components, services and the dependency injection system
\end{itemize}

Below is a list of Angular's core Metadata classes categorized under
directives/components, pipes and di.

\textbf{Directive/component Meta-data}

\begin{itemize}
\item
  \href{https://angular.io/docs/ts/latest/api/core/ComponentMetadata-class.html}{Component}:
  used to define a component

  \begin{itemize}
  \tightlist
  \item
    \href{https://angular.io/docs/ts/latest/api/core/ViewMetadata-class.html}{View}:
    used to define the template for a component
  \item
    \href{https://angular.io/docs/ts/latest/api/core/ViewChildMetadata-class.html}{ViewChild}:
    used to configure a view query
  \item
    \href{https://angular.io/docs/ts/latest/api/core/ViewChildrenMetadata-class.html}{ViewChildren}:
    used to configure a view query
  \end{itemize}
\item
  \href{https://angular.io/docs/ts/latest/api/core/DirectiveMetadata-class.html}{Directive}:
  used to define a directive

  \begin{itemize}
  \tightlist
  \item
    \href{https://angular.io/docs/ts/latest/api/core/AttributeMetadata-class.html}{Attribute}
    used to grab the value of an attribute on an element hosting a
    directive
  \item
    \href{https://angular.io/docs/ts/latest/api/core/ContentChildMetadata-class.html}{ContentChild}:
    used to configure a content query
  \item
    \href{https://angular.io/docs/ts/latest/api/core/ContentChildrenMetadata-class.html}{ContentChildren}:
    used to configure a content query
  \item
    \href{https://angular.io/docs/ts/latest/api/core/InputMetadata-class.html}{Input}:
    used to define the input to a directive/component
  \item
    \href{https://angular.io/docs/ts/latest/api/core/OutputMetadata-class.html}{Output}:
    used to define the output events of a directive/component
  \item
    \href{https://angular.io/docs/ts/latest/api/core/HostBindingMetadata-class.html}{HostBinding}:
    used to declare a host property binding
  \item
    \href{https://angular.io/docs/ts/latest/api/core/HostListenerMetadata-class.html}{HostListener}:
    used to declare a host listener
  \end{itemize}
\end{itemize}

\textbf{Pipes}

\begin{itemize}
\tightlist
\item
  \href{https://angular.io/docs/ts/latest/api/core/PipeMetadata-class.html}{Pipe}:
  used to declare reusable pipe function
\end{itemize}

\textbf{DI}

\begin{itemize}
\tightlist
\item
  \href{https://angular.io/docs/ts/latest/api/core/InjectMetadata-class.html}{Inject}:
  parameter metadata that specifies a dependency.
\item
  \href{https://angular.io/docs/ts/latest/api/core/InjectableMetadata-class.html}{Injectable}:
  a marker metadata that marks a class as available to Injector for
  creation.
\item
  \href{https://angular.io/docs/ts/latest/api/core/HostMetadata-class.html}{Host}:
  Specifies that an injector should retrieve a dependency from any
  injector until reaching the closest host.
\item
  \href{https://angular.io/docs/ts/latest/api/core/OptionalMetadata-class.html}{Optional}:
  parameter metadata that marks a dependency as optional
\item
  \href{https://angular.io/docs/ts/latest/api/core/SelfMetadata-class.html}{Self}:
  Specifies that an Injector should retrieve a dependency only from
  itself.
\item
  \href{https://angular.io/docs/ts/latest/api/core/SkipSelfMetadata-class.html}{SkipSelf}:
  Specifies that the dependency resolution should start from the parent
  injector.
\item
  \href{https://angular.io/docs/ts/latest/api/core/QueryMetadata-class.html}{Query}:
  Declares an injectable parameter to be a live list of directives or
  variable bindings from the content children of a directive.
\item
  \href{https://angular.io/docs/ts/latest/api/core/ViewQueryMetadata-class.html}{ViewQuery}:
  Similar to \texttt{QueryMetadata}, but querying the component view,
  instead of the content children.
\end{itemize}

\subsection{Dependency Injection}\label{dependency-injection}

Dependency Injection is a coding pattern in which a class receives its
dependencies from external sources rather than creating them itself. In
order to achieve Dependency Injection we need a Dependency
InjectionFramework to handle the dependencies for us. Using a DI
framework, you simply ask for a class from the injector instead of
worrying about the dependencies inside the class itself.

Angular has a standalone module that handles Dependency Injection. This
framework can also be used in non-Angular applications to handle
Dependency Injection.

\subsection{Services and Providers}\label{services-and-providers}

\begin{itemize}
\item
  A service is nothing more than a class in Angular 2. It remains
  nothing more than a class until we register it with the Angular
  injector.
\item
  When you bootstrap your app, Angular creates an injector on the fly
  that can inject services and other dependencies throughout the app.
\item
  You can register the service or the dependencies during when
  bootstrapping the app or when defining a component.
\item
  If you have a class called \texttt{MyService}, you can register it
  with the Injector and then you can inject it everywhere:

\begin{Shaded}
\begin{Highlighting}[numbers=left,,]
\FunctionTok{bootstrap}\NormalTok{(App, [MyService]); }\CommentTok{// second param is an array of providers}
\end{Highlighting}
\end{Shaded}
\item
  Providers is a way to specify what services are available inside the
  component in a hierarchical fashion.
\item
  A provider can be a class, a value or a factory.
\item
  Providers create the instances of the things that we ask the injector
  to inject.
\item
  \texttt{{[}SomeService{]};} is short for
  \texttt{{[}provide(SomeService,\ \{useClass:SomeService\}){]};} where
  the first param is the token, and the second is the definition object.
\item
  A simple object can be passed to the Injector to create a Value
  Provider:

\begin{Shaded}
\begin{Highlighting}[numbers=left,,]
\FunctionTok{beforeEachProviders}\NormalTok{(() => \{}
  \NormalTok{let someService = \{ getData: () => [] \};}
  \CommentTok{// using `useValue` instead of `useClass`}
  \KeywordTok{return} \NormalTok{[ }\FunctionTok{provide}\NormalTok{(SomeSvc, \{useValue: someService\}) ];}
\NormalTok{\});}
\end{Highlighting}
\end{Shaded}
\item
  You can also use a factory as a provider.
\item
  You can use a factory function that creates a properly configured
  Service:

\begin{Shaded}
\begin{Highlighting}[numbers=left,,]
\NormalTok{let myServiceFactory = (dx: DepX, dy: DepY) => \{}
  \KeywordTok{return} \KeywordTok{new} \FunctionTok{MyService}\NormalTok{(dx, dy.}\FunctionTok{value}\NormalTok{);}
\NormalTok{\}}

\CommentTok{// provider definition object.}
\NormalTok{let myServiceDefinition = \{}
   \NormalTok{useFactory: myServiceFactory,}
   \NormalTok{deps: [DepX, DepY]}
\NormalTok{\};}

\CommentTok{// create provider and bootstrap}
\NormalTok{let myServiceProvider = }\FunctionTok{provide}\NormalTok{(MyService, myServiceDefinition);}
\FunctionTok{bootstrap}\NormalTok{(AppComponent, [myServiceProvider, DepX, DepY]);}
\end{Highlighting}
\end{Shaded}
\item
  Defining object dependencies is simple. You can make a plain
  JavaScript object available for injection using a string-based token
  and the \texttt{@Inject} decorator:

\begin{Shaded}
\begin{Highlighting}[numbers=left,,]
\NormalTok{var myObj = \{\};}

\FunctionTok{bootstrap}\NormalTok{(AppComponent, [}
  \FunctionTok{provide}\NormalTok{('coolObjToken', \{useValue: myObj\})}
\NormalTok{]);}

\CommentTok{// and you can inject it to a component}

\KeywordTok{import} \NormalTok{\{Inject\} from 'angular2/core'}
\FunctionTok{constructor}\NormalTok{(dx: DepX, }\FunctionTok{@Inject}\NormalTok{('coolObjToken') config)}
\end{Highlighting}
\end{Shaded}
\end{itemize}

\subsection{Data Modeling and State}\label{data-modeling-and-state}

\begin{itemize}
\tightlist
\item
  Angular is flexible and doesn't prescribe a recipe for managing data
  in your apps
\item
  Since observables are integrated into Angular, you can take advantage
  of observables to manage data and state
\item
  You ca use services to manage streams that emit models
\item
  Components can subscribe to the streams maintained by services and
  render accordingly.

  \begin{itemize}
  \tightlist
  \item
    For example, you can have a service for a Todo app that contains a
    stream of todos and a \texttt{ListComponent} can listen for todos
    and render when a new task is added.
  \item
    You may have another component that listens for the user that has
    been assigned to a task provided by a service.
  \end{itemize}
\item
  The steps for creating different parts of an app can be summarized in
  three steps:

  \begin{itemize}
  \tightlist
  \item
    Defining a Model using a class
  \item
    Defining the service
  \item
    Defining the component
  \end{itemize}
\end{itemize}

\subsection{Observables}\label{observables}

\begin{itemize}
\item
  Angular embraces observables using the RxJS library.
\item
  Observables emit events and observers observe observables.
\item
  An observer \emph{subscribes} to events emitted from an observable.
\item
  RxJS has an object called \emph{subject} that can be used both as an
  observer or an observable. \emph{Subject} can be imported from
  \texttt{RxJS} very easily:

\begin{Shaded}
\begin{Highlighting}[numbers=left,,]
\KeywordTok{import \{Subject\} from 'rxjs/Subject';}
\end{Highlighting}
\end{Shaded}
\item
  A subscription can be canceled by calling the \texttt{unsubscribe}
  method.
\end{itemize}

\subsection{Angular Router}\label{angular-router}

Angular has a stand-alone module responsible for handling routing.

\section{Angular Topics in Depth}\label{angular-topics-in-depth}

Let's deep dive into Angular concepts.

\subsection{Components in Depth}\label{components-in-depth}

\begin{itemize}
\tightlist
\item
  A component declares a reusable building block of an app
\item
  A TypeScript class is used to define a component coupled with the
  \texttt{@component} decorator
\end{itemize}

\subsubsection{Components Options}\label{components-options}

The \texttt{@component} decorator defines the following:

\begin{itemize}
\item
  selector: \texttt{string} value defining the css selector targeting an
  html element
\item
  inputs: \texttt{array\ of\ string} values defining the inputs to the
  component
\item
  outputs: \texttt{array\ of\ string} values defining the output of the
  component
\item
  properties: \texttt{array\ of\ string} values defining the properties
\item
  events: \texttt{array\ of\ string} values defining the events
\item
  host?: \{{[}`string'{]}: `string'\},
\item
  providers: \texttt{array\ of\ objects} defining the providers for the
  component
\item
  exportAs: \texttt{string} value defining the exported value
\item
  moduleId: \texttt{string} value defining the module id
\item
  viewProviders: \texttt{array\ of\ objects} defining the providers for
  the view
\item
  queries: \{{[}key: string{]}: any\},
\item
  changeDetection: \texttt{ChangeDetectionStrategy} object defining the
  strategy for detecting changes:

  \begin{itemize}
  \tightlist
  \item
    \texttt{ChangeDetectionStrategy.Default}: sets detector mode to
    \texttt{CheckAlways}
  \item
    \texttt{ChangeDetectionStrategy.OnPush}: sets detector mode to
    \texttt{CheckOnce}
  \item
    \texttt{ChangeDetectionStrategy.Detached}: change detector sub tree
    is not a part of the main tree and should be skipped
  \item
    \texttt{ChangeDetectionStrategy.CheckAlways}: after calling
    detectChanges the mode of the change detector will remain
    \texttt{CheckAlways}
  \item
    \texttt{ChangeDetectionStrategy.Checked}: change detector should be
    skipped until its mode changes to \texttt{CheckOnce}
  \item
    \texttt{ChangeDetectionStrategy.CheckOnce}: after calling
    detectChanges the mode of the change detector will become
    \texttt{Checked}
  \end{itemize}
\item
  templateUrl: \texttt{string} value for the url path to the template
\item
  template: \texttt{string} value for the template
\item
  styleUrls: \texttt{array\ of\ string} values defining url paths to css
  files
\item
  styles: \texttt{array\ of\ string} values defining css styles:

  \begin{itemize}
  \tightlist
  \item
    styles: {[}`.myclass \{ color: \#000;\}'{]},
  \end{itemize}
\item
  directives: \texttt{array} of directives used in the component
\item
  pipes: \texttt{array} of pipes used in the component
\item
  encapsulation: \texttt{ViewEncapsulation} value that defines template
  and style encapsulation options:

  \begin{itemize}
  \tightlist
  \item
    \texttt{ViewEncapsulation.None}: means do not provide any style
    encapsulation
  \item
    \texttt{ViewEncapsulation.Emulated}: No Shadow DOM but style
    encapsulation emulation using extra attributes on the DOM (default
    method)
  \item
    \texttt{ViewEncapsulation.Native}: means provide native shadow DOM
    encapsulation and styles appear in component's template inside the
    shadow root.
  \end{itemize}
\end{itemize}

\end{document}
